

With the general definition of valuation over finite and infinite words in
place, we seek to define a suitable form for valuation over sets or languages,
or even to define languages based on the valuation function instead of the
classical boolean setup.

\sankalp{Build up to the current form}

We propose the following form for the valuation of a formula over a set of words:

\begin{equation}
  V(\varphi, S) = \min_{w \in S} V(\varphi, w) - \max_{w' \in \bar{S}} V(\varphi, w')
\end{equation}

Philosophically, this is the difference of the minimum score required to
'accept' a word into the set, and the maximum score with which it can be
'rejected' from it. This idea of 'acceptance' equipped onto a set is reminiscent
of the idea of languages. In fact, an alternative definition of a language may
be proposed based on this formalism.

\begin{comment}

\begin{definition}[New Languages --- new name?]
  The \emph{language} $l(\varphi)$ of an LTL formulae $\varphi$ is the set S which
  maximizes $V(\varphi, S)$.
\end{definition}

Consider $V$ as the classical boolean valuation function used in LTL. In this
case, this new definition of a language coincides exactly with the classical
case. 

\begin{example}
  Consider 
  \begin{equation}
    V(\varphi, w) = \begin{cases}
      1  & \textnormal{iff } w \models \varphi \\
      0  & \textnormal{otherwise.}
    \end{cases}
  \end{equation}

    In this case, the set $S = \{w | w \models \varphi\}$ maximizes $V(\varphi,
    \cdot)$ with value 1. Suppose not, then there exists a set $S'$ such that
    $V(\varphi, S') > V(\varphi, S)$ and $S \not = S'$. Then,

  \begin{equation}
    V(\varphi, S') = \min_{w \in S'} V(\varphi, w) - \max_{w' \in \bar{S'}} V(\varphi, w')
  \end{equation}

  with $V(\varphi, w) \geq 0\; \forall\; w$. Since $S \not = S'$, there must
  exist $w \in \bar{S}$ such that $w \in S'$, or $w' \in S$ such that $w' \in
  \bar{S'}$. In either case, $V(\varphi, S') < 1$ trivially, and we have a
  contradiction.

\end{example}

\end{comment}