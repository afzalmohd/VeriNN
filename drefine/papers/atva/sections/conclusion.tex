We have presented a novel cegar-based approach. Our approach comprises two parts. One part finds the causes of spuriousness, 
while the other part refine  the information found in the first part. 
Experimental evaluation shows that we outperform related refinement techniques, in terms of efficiency and effectivity. 
We also are able to verify several benchmarks that are beyond state of the art solvers, highly optimized solvers. Our experiments indicate when our technique can be useful and valuable as part of the portfolio of techniques for scalability of robustness verification. As futurework, we plan to extend our technique/tool to make it independent of \deeppoly{} and applicable with other 
abstraction based techniques and tools. %We believe this could lead to more benchmarks being solved overall.
 %Such that we can improve the results on top of the state of the arts.  

% Our experiments further emphasize that the DREFINE technique we use is orthogonal to most other techniques. 
% Specifically, we utilize counterexamples to identify the intrinsic cause of spuriousness, 
% such as marked or branching neurons, whereas existing methods do not analyze counterexamples to identify the neurons to branch. 
% Consequently, it is challenging to identify the exact patterns or features present in the unique benchmarks that 
% \drefine{} solves and that current state-of-the-art methods cannot solve.

% The neural network verification currently works only on small networks. However, similar to software verification, 
% where each paper contributed a small amount towards practical software verification technology, 
% we expect to make gradual progress towards practical neural network verification technology. 
% The process of discovering new methods to solve the problem and identifying which ones will lead 
% to a push-button technology is hard to predict. 
% However, as with software verification, we may develop techniques that can solve sub-problems or neighboring problems, 
% which can make significant contributions to the area.
% Experience in SAT-solving suggests that one stack of techniques is unlikely to solve all instances. 
% Therefore, we need portfolio solvers, and we contend that our technique contributes to building such a solver.

