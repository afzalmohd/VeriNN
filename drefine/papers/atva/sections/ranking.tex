%

\begin{comment}

First, we would like to understand the qualities that make up a 'good'
valuation. To that end, consider a formula $\varphi$, and the $\omega$-regular
language $L(\varphi)$ accepted by it. Given this language as a sample to work over,
perhaps we would like to infer $\varphi$. A 'good' valuation function in this
context would be one which can use this data (which contains \emph{all} the
patterns described by $\varphi$) to differentiate and choose $\varphi$ out of an
arbitrary selection of LTL formulae. We now formalize this.

\sankalp{Is this a theorem? Doesn't feel right, better terminology to be used. 
It's just a condition we're imposing.}

\begin{theorem}
  A valuation function $V$, evaluating formulae over the $\omega$-regular
  language $L(\varphi)$ generated by an LTL formula $\varphi \in \ltf$, i.e.
  $V(\cdot, L(\varphi))$ must differentiate and prefer $\varphi$ over all
  formulae $\varpsi \in \ltf$ not equivalent to it, i.e., $V(\varphi,
  L(\varphi)) > V(\varpsi, L(\varphi)) \;\forall \varpsi \in \ltf$ where
  $\varpsi \not \equiv \varphi$.

  Consider an arbitrary $\varpsi \in \ltf$. We only have the information about
  $\varpsi$ contained in $L(\varphi)$, i.e., $L(\varpsi) \intersection
  L(\varphi)$. There are two cases to consider for this information:
  \begin{itemize}
    \item $L(\varphi) \backslash \;(L(\varpsi) \intersection L(\varphi)) \not =
          \emptyset$ This case is trivial, as $L(\varphi)$ contains information
          to distinguish between $\varphi$ and $\varpsi$ by way of infinite
          words which satisfy one but not the other.

    \item $L(\varphi) \backslash \;(L(\varpsi) \intersection L(\varphi)) =
          \emptyset$ This is the case where $L(\varphi) \subset L(\varpsi)$.
          There are no obvious differentiators of the pair anymore, so they must
          be differentiated based on score, and per our requirement, we must
          have $\varphi$ gaining a better score. 
          
          \sankalp{bypass the second part of section, move to implication definition cleanly}

          \sankalp{add lemma showing inclusion is implication?} In this case, we
          must have $\varphi \Rightarrow \varpsi$. To differentiate between the
          two, we now require the valuation function to be able to differentiate
          between the two \sankalp{add how?}.

          We claim that there are only
          two scenarios where this may occur:

          \begin{enumerate}
            \item $\varphi = \varpsi \land \chi$ for some $\chi \in \ltf$, or
            $\varphi = \globally \varpsi$ \\

                  Intuitively, this is the case where $L(\varpsi)$ was shrunken
                  by composition to construct $L(\varphi)$. Since we must have
                  $V(\varphi) > V(\varpsi)$, we get contraints for the valuation
                  function $V$ itself:
                  \begin{enumerate}
                    \item $V(\varpsi \land \chi) > V(\varpsi), V(\chi) \;
                    \forall \varpsi, \chi \in \ltf$
                    \item $V(\globally \varpsi) > V(\varpsi) \; \forall \varpsi
                    \in \ltf$
                    \newcounter{constraints}
                    \setcounter{constraints}{\value{enumii}}
                  \end{enumerate}

            \item $\varpsi = \varphi \lor \chi$ for some $\chi \in \ltf$, or
            $\varpsi = \finally \varphi$ \\
                  Intuitively, this is the case where $L(\varphi)$ was enlarged
                  by composition to construct $L(\varpsi)$. Since we must have
                  $V(\varphi) > V(\varpsi)$, we get contraints for the valuation
                  function $V$ again:

                  \begin{enumerate}
                    \setcounter{enumii}{\value{constraints}}
                    \item $V(\varpsi \lor \chi) < V(\varpsi), V(\chi) \; \forall
                    \varpsi, \chi \in \ltf$
                    \item $V(\finally \varpsi) < V(\varpsi) \; \forall \varpsi
                    \in \ltf$
                  \end{enumerate}
          \end{enumerate}
  \end{itemize}

\end{theorem}

\end{comment}
