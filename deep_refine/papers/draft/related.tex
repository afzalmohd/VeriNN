 The papers closest to our work are  
\cite{lemieux2015generalTexada} and \cite{ltlFMCAD18}.  The focus of
\cite{ltlFMCAD18} is to produce the minimal formula which is consistent with a
rational sample irrespective of the expressiveness, while
\cite{lemieux2015generalTexada} requires a user defined input template of the
LTL formula which they would like to satisfy.  Both work with infinite traces :
\cite{ltlFMCAD18} takes rational traces in the form $uv^{\omega}$ where $u,v$
are typically words of length $\sim 10$, while \cite{lemieux2015generalTexada}
mines specifications from finite traces, and appends them with an infinite
sequence of terminal events.  
 \cite{DBLP:conf/otm/AalstBD05t42} considers finite traces, and develops a LTL
checker which takes an event log and a LTL property and verifies if the observed
behaviour matches some bad behaviour. The papers
\cite{989799t19,10.1007/3-540-46002-0_24t20} look at the application of
monitoring the execution of Java programs, and check LTL formulae on finite
traces of these programs.   
In \cite{lo2012mining33}, the authors focus on mining quantified temporal rules
which help in establishing data flow analysis between variables in a program,
while in \cite{weimer2005mining46} software bugs are exposed  using a mining
algorithm, especially for control flow paths. \cite{agrawal1998miningt9} looks
at process mining in the context of workflow management. The tool PISA
\cite{zur2000workflowt27} is developed to extract the performance matrix from
workflow logs, while a declarative language  is developed to formulate
workflow-log properties in \cite{6951474t5}. Tool Synoptic
\cite{beschastnikh2011leveraging7} on the other hand, follows  a different
approach and takes event logs and regular expressions as input and produces a
model that satisfies a temporal invariant which has been mined from the trace.
%Finally, many  recent works
%\cite{beschastnikh2014inferring6,beschastnikh2011leveraging7,lo2009automatic32,walkinshaw2008inferring44,lemieux2015generalTexada},
%rely on temporal properties to infer various types of properties from source
%code, log files, and execution traces. 
