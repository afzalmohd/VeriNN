In this section, we present some basic definitions, starting with a neural network.
\begin{df}
  \label{def:net}
    A neural network $N = (Neurons, Layers, Edges, W, B, Type)$ is a 6-tuple, where
    \begin{itemize}
        \item $Neurons$ is the set of neurons in $N$,
        \item $Layers = \{l_0,...,l_k\}$ is an indexed partition of $Neurons$,
        % \item $l_i \in Layers$ represents the $i^{th}$ layer in network $N$. 
        % \item $|l_i|$ represents the number of neurons in layer $l_i$. 
        % \item 
        \item $ Edges \subseteq \Union_{i=1}^{k} l_{i-1} \times l_{i}$ is a set of edges linking neurons on consecutive layers,
        \item $W : Edges \mapsto \reals$ is a weight function on edges,
        \item $B : Neurons \mapsto \reals$ is a bias function on neurons,
        \item $Type : Layers \mapsto \{\affine{}, \relu{}\}$ defines type of neurons on each layer.
    \end{itemize}
\end{df}

A neural network is a collection of layers $l_0, l_1, l_2, ... l_k$, where $k$ represents the number of layers. Each layer contains neurons that are also indexed, with $n_{ij}$ denoting the $j$th neuron of layer $l_i$. We call $l_0$ and $l_k$ the {\em input} and {\em output layers} respectively, and all other layers as {\em hidden layers}. In our presentation, we assume separate layers for the activation functions. Though there are different kinds of activations, we focus only on $\relu${}, hence each layer can either be  $\affine${} or $\relu${} layer. The definition of $W$ and $B$ applies only to the $\affine${} layer. Without loss of generality, we assume that the output layer is an \affine{} layer (we can always append an identity \affine{} layer), and layers $l_1, l_3, l_5, ..., l_k$ to be the \affine{} layers, layers $l_2, l_4, l_6, ..., l_{k-1}$ to be the \relu{} layers. If $Type_i = \relu{}$, then $|l_{i-1}| = |l_{i}|$. We extend the weight function from edges to layers using matrix $W_i \in \reals^{|l_i|\times|l_{i-1}|}$ that represents the weight for layer $l_i$, s.t.,
$$
W_i[t_1, t_2] = 
\begin{cases}
  W(e) & e=(n_{(i-1)t_2}, n_{it_1}) \in Edges,\\
  0 & \text{otherwise.}\\
\end{cases}
$$
We also write matrix $B_i \in \reals^{|l_i|\times 1}$ to denote the bias matrix for layer $l_i$. The entry $B_i[t,0] = B(n_{it})$, where $n_{it} \in Neurons$. 

% There is a one-to-one mapping between an affine layer $l_{i-1}$ and a $\relu${} layer $l_i$. 
% If a layer $l_i$ is $\relu${} layer, then the input neuron of $n_{ij}$ will only be the neuron $n_{(i-1)j}$. 


To define the semantics of $N$, we will use vectors ${val_i} = [val_{i1}, val_{i2}, ... val_{i|l_i|}]$ that represent the values of each neuron in the layer $l_i$. Let $f_i$ be a function that computes the output vector of values at layer $i$ using the values at layer $i-1$ as $val_i = f_i(val_{i-1})$. For each type the layer the functions are defined as follows: if $Type_i = \affine$, then $f_{i}(val_{i-1}) = W_i * val_{i-1} + B_i$; if $Type_i = \relu$, then $f_{i}(val_{i-1})_j =  max(val_{{(i-1)}j},0)$. Then, the semantics of a neural network $N$ is a function (we abuse notation and also denote this function as $N$) which takes an input, an $|l_0|$-dimensional vector of reals and gives as output an $|l_k|$-dimensional vector of reals, as a composition of functions $f_k \circ ... \circ f_1$. Thus, for an input $v\in \mathbb{R}^{|l_0|}$, we write its value computed by $N$ at layer $i$ as $val^v_{i}=f_i \circ ... \circ f_1(v)$.

Let us define 
$LinExpr = \{ w_0 + \sum_{i} w_i x_i | w_i \in \reals \text{ and } x_i \text{ is a real variable} \}$
and
$LinConstr = \{expr \text{ op } 0 | expr \in LinExpr \land op \in \{\leq, = \}\}$.
A {\em predicate} is a Boolean combination of $LinConstr$.
We use real variable $x_{ij}$ to represent values of $n_{ij}$ in the predicates.
Let $P$ and $Q$ be predicates over input and output layers respectively.
A {\em verification query} is a triple $\langle N, P, Q \rangle$.
We need to prove that for each input ${v}$,
if ${v} \models P$, $N({v}) \models Q$.
We assume $P$ has the form
% Let us say $P$ is a predicate on input layer's variables
$\Land_{i=1}^{|l_0|}lb_{0i} \leq x_{0i} \leq ub_{0i}$, where $lb_{0i}$, $ub_{0i}$ are lower and upper bounds respectively for a neuron $n_{0i}$.


% The value of ${val_i}$ is computed by the weighted sum of the
% previous layer's values($W_i * V_{i-1} + B_i$)
% if $l_i$ is an affine layer, otherwise ${val_i}$ is computed by the $\relu${} function. 
% A function $y = max(0,x)$ is a $\relu${} function that takes an argument $x$ as input and returns the
% same value $x$ as output if $x$ is non-negative otherwise return the value 0. 


% where each function $f_i$ represents either the linear combinations of the previous layer's
% output or an activation function.

% Let us say $n_{ij} \in N.Neurons$ represent 
% the $i^{th}$ neuron of layer $l_j$,

% Let us say $val_i^{x}$ represent the point on layer $l_i$ if neural network executes on input $x$. 

% For any vector ${v}$, $v_i$ represents it's $i^{th}$ value.  

% Let $\reals$ be the set of real numbers.
% Let $x_{\alpha}$ are unbounded set of real variables, where
% $\alpha$ is arbitrary index for variables.

% \begin{df}
%   \label{def:linexpr}
% \end{df}
  
% \begin{df}
%     \label{def:linconstr}
% \end{df}



\subsection{DeepPoly}
\label{sec:deeppoly}

We develop our abstract refinement approaches on top of abstraction based method
\deeppoly{}~\cite{singh2019abstract}, which uses a combination of
well-understood polyhedra~\cite{cousot1978automatic} and box~\cite{cousot1977abstract} abstract domain.
The abstraction maintains upper and lower linear expressions as well as upper and lower bounds for each neuron.
The variables appearing in upper and lower expressions are only from the predecessor layer.
Formally, we define the abstraction as follows. 
% \todo{Add some intuition}
% Globally \texttt{DeepPoly} forms a polyhedron.
% Experimentally, it has better precision in comparison to Box~\cite{} and Zonotope~\cite{}.
% Deeppoly has the abstract transformer of various types of layers and activation functions.


\begin{df}
    For a neuron $n$,
    an abstract constraint $A(n) = (lb,ub, lexpr, uexpr)$ is a tuple, where
    $lb \in \reals$ is lower bound on the value of $n$,
    $ub \in \reals$ is the upper bound on the value of  $n$,
    $lexpr \in LinExpr$ is the expression for the lower bound, and
    $uexpr \in LinExpr$ is the expression for the upper bound.
\end{df}

In \deeppoly{}, we compute the abstraction $A$ as follows.

% The abstract constraint $A$ is generated by the tool deeppoly~\cite{} as explained in subsection~\ref*{sec:deeppoly}. 

\begin{itemize}
\item If $Type_i= \affine$, we set 
  $A(x_{ij}).lexpr := A(x_{ij}).uexpr := \sum_{t=1}^{|l_{i-1}|} W_i[j,t]*x_{(i-1)t} + B_i[j,0]$.
  We compute $A(x_{ij}).lb$ and $A(x_{ij}).ub$ by back substituting
  the variables in $A(x_{ij}).lexpr$ and $A(x_{ij}).uexpr$ respectively up to input layer.
  Since $P$ of the verification query has lower and upper bounds of the input layer,
  we can compute the bounds for $x_{ij}$.

\item If $Type_i= \relu$ and $y=\relu(x)$, we consider the following three cases: %\todo{I tried to simplify the notation, please have a look}:
  \begin{enumerate}
  \item If $A(x).lb \geq 0$ then $\relu$ is in active phase and
    $A(y).lexpr := A(y).uexpr := x$,
    and $A(y).lb := A(x).lb$ and $A(y).ub := A(x).ub$
  \item If $A(x).ub \leq 0$ then $\relu$ is in passive phase and
    $A(y).lexpr := A(y).uexpr := 0$, 
    and $A(y).lb := A(y).ub := 0$.
  \item  If $A(x).lb < 0$ and $A(x).ub > 0$,
    the behavior of $\relu$ is uncertain, and we need to apply
    over-approximation. We set $A(y).uexpr := u(x - l) / (u - l)$, 
    where $u = A(x).ub \text{ and } l = A(x).lb$.
    And $A(y).lexpr := \lambda . x$, where $\lambda \in \{0,1\}$. 
    We can choose any value of $\lambda$ dynamically.
    We compute $A(y).lb$ and $A(y).ub$ by doing the back-subtitution similar to the \affine{} layer's neuron.
  \end{enumerate}
%} 



% \item If $Type_i= \relu$, we consider the following three cases:
%   \begin{enumerate}
%   \item If $A(x_{(i-1)j}).lb \geq 0$ then $\relu$ is in active phase and
%     $A(x_{ij}).lexpr := A(x_{ij}).uexpr := A(x_{(i-1)j}).lexpr $,
%     and $A(x_{ij}).lb := A(x_{(i-1)j}).lb$ and $A(x_{ij}).ub := A(x_{(i-1)j}).ub$
%   \item If $A(x_{(i-1)j}).ub \leq 0$ then $\relu$ is in passive phase and
%     $A(x_{ij}).lexpr := A(x_{ij}).uexpr := 0$, 
%     and $A(x_{ij}).lb := A(x_{ij}).ub := 0$.
%   \item  If $A(x_{(i-1)j}).lb < 0$ and $A(x_{(i-1)j}).ub > 0$,
%     the behavior of $\relu$ is uncertain, and we need to apply
%     over-approximation. We set $A(x_{ij}).uexpr := u(x_{(i-1)j} - l) / (u - l)$, 
%     where $u = A(x_{(i-1)j}).ub \text{ and } l = A(x_{(i-1)j}).lb$.
%     And $A(x_{ij}).lexpr := \lambda . x_{(i-1)j}$, where $\lambda \in \{0,1\}$. 
%     We can choose any value of $\lambda$ in a run.
%     We compute $A(x_{ij}).lb$ and $A(x_{ij}).ub$ in the same way as
%     in the case of \affine{} neurons.
%     % They compute the value of $A(x_{ij}).lb$ and $A(x_{ij}).ub$ 
%     % by back substituting the variables in $A(x_{ij}).lexpr$ and
%     % $A(x_{ij}).uexpr$ respectively up to the input layer. 
%   \end{enumerate} 
\end{itemize}

The constraints for an \affine{} neuron are exact because it is just an \affine{}
transformation of input neurons. 
The constraints for a \relu{} neuron are also exact if the \relu{} is either
in the active or passive phase. 
The constraints for \relu{} are over-approximated if the behavior of \relu{}
is uncertain. Although we may compute exact 
constraints for this case, but the constraints will be arbitrary polyhedron,
which are expensive to compute.
The \deeppoly{} abstraction finds a balance between precision and efficiency.

For the verification query $\langle N, P, Q \rangle$, we check if
$\lnot Q \land \Land_{j=1}^{|l_k|}lb_{kj} \leq x_{kj} \leq ub_{kj}$
are satisfiable.
%
If the formula is unsatisfied then we have proven the query successfully.
Otherwise, \deeppoly{} fails to prove the query.




\subsection{Solver}
\label{sec:solver}

% In our algorithms, we will be needing a optimizer and a solver.
% To solve the verification queries,
% we are using Gurobi(v9.1)~\cite{gurobioptimizer} to optimize the query or to check the satisfiability. 
% The function \texttt{checkSAT} in algorithm \ref{algo:verif1}, 
% takes a Boolean formula as input and returns \texttt{SAT} or \texttt{UNSAT}. 
% In the case of \texttt{UNSAT}, it returns the irreducible inconsistent subsystem (\texttt{iis}). 
% The \texttt{iis} is a subset of constraints, which are \texttt{UNSAT}, 
% but if a single constraint removed from \texttt{iis} then it becomes \texttt{SAT}.
% The function \texttt{maxsoftConstr} in the algorithm \ref{algo:refine2} takes
% two arguments as input \texttt{hardConstr} and \texttt{softConstrs}. 
% The \texttt{hardConstr} is a Boolean formula of constraints,
% and \texttt{softConstrs} is a set of constraints. 
% The function \texttt{maxsoftConstr} satisfies the maximum number of constraints
% in \texttt{softConstrs} while satisfying the \texttt{hardConstr}. 
% The function \texttt{maxsoftConstr} returns \texttt{SAT} with the set of
% constraints satisfied in \texttt{softConstr}, or returns
% \texttt{UNSAT} if \texttt{hardConstr} fails to satisfy.
% We use an optimization query in gurobi to implement \texttt{maxsoftConstr} function.

In our algorithms, we are using two major calls \checksat{} and \maxsat{}.
The function \checksat{} in algorithm \ref{algo:verif1}, 
takes a quantifier-free formula as input and returns \sat{} or \unsat{}. 
% In the case of \texttt{UNSAT}, it returns the irreducible inconsistent subsystem (\texttt{iis}). 
% The \texttt{iis} is a subset of constraints, which are \texttt{UNSAT}, 
% but if a single constraint is removed from \texttt{iis} then it becomes \texttt{SAT}.
The function \maxsat{} in the algorithm \ref{algo:refine2} takes
two arguments as input \hardconstr{} and \softconstr{}. 
The \hardconstr{} is a Boolean formula of constraints,
and \softconstr{} is a set of constraints. 
The function \maxsat{} satisfies the maximum number of constraints
in \softconstr{} while satisfying the \hardconstr{}. 
The function \maxsat{} returns \sat{} with the set of
constraints satisfied in \softconstr{}, or returns
\unsat{} if \hardconstr{} fails to satisfy.
We are using Gurobi(v9.1)~\cite{gurobioptimizer} to implement 
both \checksat{} and \maxsat{} functions. 





%%% Local Variables:
%%% mode: latex
%%% TeX-master: "../main"
%%% End:
