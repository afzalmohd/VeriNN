% one paragraph per point
% about 2 pages 



% Current state

% Zoom into deeppoly

% your technique (2 para )


% Experiments:
% tools (why those tools), benchmark, results

% Layout of the rest of the paper


% Why this problem?
In the recent years, neural networks are being increasingly used in safety-critical systems such as autonomous vehicles, medical diagnosis, and speech recognition~\cite{bojarski2016end,amato2013artificial,hinton2012deep}. Since the neural networks are trained using data, they may not give the expected output for each unseen input. It is now well-established (see e.g., Goodfellow \cite{goodfellow2014explaining}) that a slight change in the input can fool the neural networks. Such errors can be hard to find/analyze/debug by inspection as these neural networks contain hundreds of thousands of non-linear nodes.

% What is the problem?
To address this problem, an entire line of research has focussed on automatic (robustness0 verification of the networks. Since automatic verification of the neural network is an NP-hard problem~\cite{?}, therefore researchers use approximations in their methods. Classically, we may divide the methods into two classes, namely complete and incomplete. The methods~\cite{lomuscio2017approach,fischetti2018deep,dutta2018output,cheng2017maximum,katz2017reluplex,katz2019marabou,ehlers2017formal,huang2017safety,wang2021beta,xu2020fast,zhang2022general,alpha} are complete methods, i.e., they explore the exact state space.
Since complete methods explore exact state space, they suffer from scalability issues on the large-scale networks. On the other hand, the abstraction based methods \cite{dvijotham2018dual}, \cite{gehr2018ai2}, \cite{singh2018fast},
 \cite{singh2018boosting}, \cite{weng2018towards}, \cite{wong2018provable}, \cite{zhang2018efficient}, \cite{zhang2018efficient} are sound and incomplete because they over approximate the state space. The incomplete methods scale well in comparison to complete methods.  The abstraction based methods suffer from imprecision, so, the methods \cite{wang2018formal,wang2018efficient,elboher2020abstraction,yang2021improving,lin2020art} refine the over-approximated state space to achieve completeness. In \cite{wang2018formal,wang2018efficient,lin2020art} the authors eliminate the spurious information by bisecting the input space on the guided dimension. The \cite{yang2021improving} works on top of \deeppoly{}~\cite{singh2019abstract}, they remove
the spurious region by taking each neuron's constraints with the negation of the property and using the  Gurobi optimizer \cite{gurobioptimizer} to refine the bounds of neurons. Elbohar et.al. \cite{elboher2020abstraction} defined the four classes of neurons based on their characteristics. At the time of abstraction, they merge each neuron into one of the four classes.  After completing the abstraction process they use the state of the complete verifier to verify  the abstract network, they go to the refinement process in case of failure.   In the refinement, process authors split the merged neurons.  In the worst case, this method may get back to the original network.  Although this work is a cegar-based approach, it also suffers from scalability issues on large-scale networks.   The work \cite{elboher2020abstraction} refinement process focuses on the structure of the networks.

Our method is top on the \deeppoly{}.  \deeppoly{} maintains a single upper and a single lower linear constraints as well as lower and upper bounds
for each neuron in the network. For an affine neuron, the upper and lower constraints are the same as an affine expression,
which is a weighted sum of the input neurons. For a $\relu${} neuron, they are constructing the upper and lower constraints
by doing the triangle approximation~\cite{singh2019abstract}. 
Triangle approximation is also explained in section~\ref{sec:deeppoly}. 
\deeppoly{} scales well on large-scale networks, but it also suffers from imprecision. 

In this paper, we remove the imprecision by doing a novel refinement, which is cegar-based. 
Whenever \deeppoly{} fails to verify the property, we collect all the linear constraints generated by it, 
with the negation of the property, and check for satisfiability, by using an \milp{}-based tool. 
If the tool return \unsat{} then we report property \texttt{verified}. Otherwise, we go to the refinement process.
We have two parts of our refinement approach, one finds the causes of spurious information 
and the second part refines the information gets in the first part. 
We have two methods to find the causes of spuriousness namely the {\em pullback} method, and the 
{\em optimization-based} method. And two methods for the refinement namely {\em splitting} based method and 
{\em MILP based}-method.    

We explain our approach with a motivating example in the next section~\ref{sec:motivation}. 
We define the notions and definitions in section~\ref{sec:model}. The section~\ref{sec:algo} contains the 
algorithm procedure of our approach and section~\ref{sec:experiments} contains the experiments. 
We present the related work and future work in sections \ref{sec:related} and \ref{sec:conclusion} respectively. 


% --- do not erase below this line ----

%%% Local Variables:
%%% mode: latex
%%% TeX-master: "../main"
%%% End:
