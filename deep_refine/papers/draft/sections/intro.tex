In recent years, neural networks are increasingly being applied in
safety-critical systems such as autonomous vehicles \cite{bojarski2016end},
medical diagnosis \cite{amato2013artificial}, and speech recognition \cite{hinton2012deep}.
Goodfellow \cite{goodfellow2014explaining} has shown that slight change in the input can fool the neural networks,
i.e., image misclassification by the change in a few pixels.
The developers find it hard to analyze/debug the neural networks because they contain hundreds of thousands of non-linear nodes.
To address this problem, developers need automatic verification of the networks.
Since automatic verification of the neural network is an NP-hard problem, therefore researchers use approximations in their methods.
We may divide the methods into two classes, namely complete and incomplete.
The methods\cite{lomuscio2017approach}, \cite{fischetti2018deep},
\cite{dutta2018output}, \cite{cheng2017maximum}, \cite{katz2017reluplex}, \cite{katz2019marabou}, 
\cite{ehlers2017formal}, \cite{huang2017safety}, \cite{wang2021beta}, \cite{xu2020fast}, \cite{zhang2022general}
are complete methods, i.e., they explore the exact state space.
Since complete methods explore exact state space, they suffers with the scalability issues on large scale network.
On the other hand, the abstraction based methods
\cite{dvijotham2018dual}, \cite{gehr2018ai2}, \cite{singh2018fast},
 \cite{singh2018boosting}, \cite{weng2018towards}, \cite{wong2018provable}, \cite{zhang2018efficient}, \cite{zhang2018efficient}
are sound and incomplete because they over approximate the state space.
The incomplete methods scale well in comparison to complete methods. 
The abstraction based methods suffer with imprecision, hence the methods
\cite{wang2018formal,wang2018efficient,elboher2020abstraction,yang2021improving,lin2020art}
refine the over-approximated state space to achieve completeness. 
Papers \cite{wang2018formal,wang2018efficient,lin2020art} eliminate the
spurious information by bisecting the input space on the guided dimension. The \cite{yang2021improving} works
on top of \deeppoly{}~\cite{singh2019abstract}, the remove
the spurious region by taking each neuron's constraints with negation of the property and use the 
Gurobi optimizer \cite{gurobioptimizer} to refine the bounds of neurons.
The \cite{elboher2020abstraction} defined the four classes of neurons based on their characteristics.
At the time of abstraction, \cite{elboher2020abstraction} merge each neuron into one of the four classes. 
After completing the abstraction process they use state of the complete verifier to verify 
the abstract network, they go to the refinement process in case of failure.  
In the refinement process \cite{elboher2020abstraction} partition one of the four classes into two subclasses. 
In the worst case, this method may get back to the original network. 
The work \cite{elboher2020abstraction} refinement process focus on the structure of the networks. 
While our work refining the abstraction based on \deeppoly{}.
We two parts of our approach, one finds the causes of spurious information and the second part refine the 
information gets in first part. We two methods to find the causes of spuriousness namely {\em pullback} method, and 
{\em optimization based} method. And two methods for the refinement namely {em splitting} based method and 
{\em MILP based} method.    

We explain our approach by a motivating example in next section~\ref*{sec:motivation}. 
We define the notions and definitions in section~\ref{sec:model}. The section~\ref{sec:algo} contains the 
algorithm process of our approach, and the section~\ref{sec:experiments} contains the experiments. 
We present the related work and future work in sections \ref{sec:related} and \ref{sec:conclusion} respectively. 


