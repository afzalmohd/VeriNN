Consider a neural network in figure~\ref{fig:motivating}, 
which has one input, one hidden and one output layer. The hidden layer separated into two layers 
\texttt{Affine} and \texttt{ReLU} layers, so total four layers shown in figure~\ref{fig:motivating}. 
Every layer contains two neurons. The neuron $x_8$ has bias $0.2$ and all the other neurons has bias $0$. 
Our goal is to verify for all possible input $x_1,x_2 \in [-1,1]$ the outputs $x_7 < x_8$ satisfy. 
Our approach is top on \deeppoly{}~\ref{deeppoly}. \deeppoly{} maintains the one upper and one lower constraint
as well as upper and lower bound for each neurons. For a neuron of affine layer the upper and lower constraint is 
same, which is the weighted sum of the input neurons i.e. $x_3$'s upper and lower constraint is $x_1+x_2$. 
For an activation neuron the upper and lower expression is computed using triangle approximation~\cite{deeppoly}, 
also briefly explained in section~\ref{sec:deeppoly}. To verify the property $x_7 < x_8$, \deeppoly{} creates a 
new expression $x_9 = x_7 - x_8$ and compute the upper bound of $x_9$. The upper bound of $x_9$ should not be greater
than $0$. \deeppoly{} computes the upper bound of $x_9$ by back substituting the expression of $x_7$ and $x_8$ 
from the previous layer, the back substitute further from the previous layer. They keep back substitute 
up to the input layer. The process of back substitution is shown in equation~\ref{eq:deeppoly}.
The upper bound of $x_9$ is computed $1.8$ which is greater than $0$, hence, the \deeppoly{} fails to verify the property. 


\begin{equation}
    \begin{aligned}
        x_9 \leq  &  x_7 - x_8 \\
        x_9 \leq  & x_5 - x_6 - 0.2 \\
        x_9 \leq  & 0.5x_3 + 1 - 0.2 \\
        x_9 \leq  & 0.5(x_1+x_2) + 1 -0.2 \\
        x_9 \leq  & 1.8
    \end{aligned}
\label{eq:deeppoly}
\end{equation}



\begin{figure}[!ht]
	\centering
	\scalebox{0.8}{\begin{tikzpicture}[
    % define styles 
    clear/.style={ 
        draw=none,
        fill=none
    },
    net/.style={
        matrix of nodes,
        nodes={ draw, circle, inner sep=3pt },
        nodes in empty cells,
        column sep=1cm,
        row sep=.1cm
    },
    >=latex
]
% define matrix mat to hold nodes
% using net as default style for cells
\matrix[net] (mat)
{
% Define layer headings
   |[clear]| \parbox{1.5cm}{\centering $\langle x_1 \geq -1$, \\ $x_1 \leq 1$, \\ $l_1 = -1$, \\ $u_1 = 1 \rangle$} 
   & |[clear]| \parbox{2cm}{\centering $\langle x_3 \geq x_1 + x_2$, \\ $x_3 \leq x_1 + x_2$, \\ $l_3 = -2$ \\ $u_3 = 2 \rangle$}
   & |[clear]| \parbox{2.2cm}{\centering $\langle x_5 \geq 0$, \\ $x_5 \leq 0.5.x_3 + 1$, \\ $l_5 = 0$, \\ $u_5 = 2 \rangle$}
   & |[clear]|  \parbox{1.5cm}{\centering $\langle x_7 \geq x_5$, \\ $x_7 \leq x_5$, \\ $l_1 = 0$, \\ $u_1 = 2 \rangle$} \\
        
$x_1$  & $x_3$  & $x_5$ & $x_7$ \\
|[clear]| & |[clear]| & |[clear]| & |[clear]| \\
|[clear]| & |[clear]| & |[clear]| & |[clear]| \\
|[clear]| & |[clear]| & |[clear]| & |[clear]| $0.2$ \\
$x_2$  & $x_4$  & $x_6$ & $x_8$ \\
|[clear]| \parbox{1.5cm}{\centering $\langle x_2 \geq -1$, \\ $x_2 \leq 1$, \\ $l_2 = -1$, \\ $u_2 = 1 \rangle$} 
& |[clear]| \parbox{2cm}{\centering $\langle x_4 \geq x_1 + x_2$, \\ $x_4 \leq x_1 + x_2$, \\ $l_4 = -2$ \\ $u_4 = 2 \rangle$}
& |[clear]| \parbox{2.2cm}{\centering $\langle x_6 \geq 0$, \\ $x_6 \leq 0.5.x_4 + 1$, \\ $l_6 = 0$, \\ $u_6 = 2 \rangle$}
& |[clear]| \parbox{1.5cm}{\centering $\langle x_8 \geq x_6$, \\ $x_8 \leq x_6$, \\ $l_1 = 0$, \\ $u_1 = 2 \rangle$} \\
};

% left most lines into input layers
\draw[<-] (mat-2-1) -- +(-1cm,0) node[pos=1, above] {$[-1,1]$};
\draw[<-] (mat-6-1) -- +(-1cm,0) node[pos=1, above] {$[-1,1]$};

\draw[->] (mat-2-1) -- (mat-2-2) node[pos=0.5, above] {$1$};
\draw[->] (mat-2-1) -- (mat-6-2) node[pos=0.25, above] {$1$};
\draw[->] (mat-6-1) -- (mat-2-2) node[pos=0.25, above] {$1$};
\draw[->] (mat-6-1) -- (mat-6-2) node[pos=0.5, above] {$1$};

\draw[->] (mat-2-2) -- (mat-2-3) node[pos=0.5, above] {$max(0,x_3)$};
\draw[->] (mat-6-2) -- (mat-6-3) node[pos=0.5, above] {$max(0,x_4)$};

\draw[->] (mat-2-3) -- (mat-2-4) node[pos=0.5, above] {$1$};
\draw[->] (mat-2-3) -- (mat-6-4) node[pos=0.25, above] {$0$};
\draw[->] (mat-6-3) -- (mat-2-4) node[pos=0.25, above] {$0$};
\draw[->] (mat-6-3) -- (mat-6-4) node[pos=0.5, above] {$1$};


%\draw[->] (mat-2-4) -- +(1cm,0) node[pos=1, above] {$[0,3]$};
%\draw[->] (mat-6-4) -- +(1cm,0) node[pos=1, above] {$[-2,2]$};



% lines from a_{i}^{1} to a_{0}^{2}
%\foreach \ai in {1,2}
%  \draw[->] (mat-\ai-2) -- (mat-6-3);
   
% right most line with Output label
%\draw[->] (mat-6-3) -- node[above] {Output} +(2cm,0);
\end{tikzpicture}
}
	\caption{Hypothetical example of neural network}
	\label{fig:motivating}
\end{figure}