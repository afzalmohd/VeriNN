Consider a neural network in figure~\ref{fig:motivating}, 
which has one input, one hidden and one output layer. The hidden layer separated into two layers 
\texttt{Affine} and \texttt{ReLU} layers, so total four layers shown in figure~\ref{fig:motivating}. 
Every layer contains two neurons. The neuron $x_8$ has bias $1$ and all the other neurons has bias $0$. 
Our goal is to verify for all possible input $x_1,x_2 \in [-1,1]$ the outputs $x_7 \leq x_8$ satisfy. 
Our approach is top on \deeppoly{}~\cite{singh2019abstract}. \deeppoly{} maintains the one upper and one lower constraint
as well as upper and lower bound for each neurons. For a neuron of affine layer the upper and lower constraint is 
same, which is the weighted sum of the input neurons i.e. $x_3$'s upper and lower constraint is $x_1+x_2$. 
For an activation neuron the upper and lower expression is computed using triangle approximation~\cite{singh2019abstract}, 
which is briefly explained in section~\ref{sec:deeppoly}. To verify the property $x_7 \leq x_8$, \deeppoly{} creates a 
new expression $x_9 = x_7 - x_8$ and compute the upper bound of $x_9$. The upper bound of $x_9$ should not be greater
than $0$. \deeppoly{} computes the upper bound of $x_9$ by back substituting the expression of $x_7$ and $x_8$ 
from the previous layer, the back substitute further from the previous layer. They keep back substitute 
up to the input layer. The process of back substitution is shown in equation~\ref{eq:deeppoly}.
After back substitution the upper bound of $x_9$ is computed $1$ which is greater than $0$, 
hence, the \deeppoly{} fails to verify the property. 


\begin{equation}
    \begin{aligned}
        x_9 \leq  &  x_7 - x_8 \\
        x_9 \leq  & x_5 - x_6 - 1 \\
        x_9 \leq  & 0.5x_3 + 1 - 1 \\
        x_9 \leq  & 0.5(x_1+x_2) \\
        x_9 \leq  & 1
    \end{aligned}
\label{eq:deeppoly}
\end{equation}

There are two main reasons of the failure of \deeppoly{}. One is it can not maintain the complete corelation 
between the neurons. In this example neurons $x_3$ and $x_4$ has the same expression $x_1+x_2$, so, they always
get the same value. But in the \deeppoly{} analysis process it may fails to get the same value. Second is the triangle
approximation on relu neurons.
We take the conjunction of upper and lower expressions of each neurons with the negation of the property
as shown in equation~\ref{eq:conjunction},
and use \milp{} solver to check the satisfiability and tackled the first issue. 
The second issue can be resolved either by splitting the 
bound at zero of the affine node or by using the exact encoding (eq~\ref{eq:reluexact}) instead of triangle approximation.
Both solutions increase the problem size exponentially in terms of relu neurons. 
The problem size will be huge if we repair every neuron of the network. 

\begin{equation}
    \begin{aligned}
        -1 \leq & x_1 \leq 1 \\ 
        -1 \leq & x_2 \leq 1 \\
         x_1 + x_2 \leq & x_3 \leq x_1 + x_2 \\
         x_1 + x_2 \leq & x_4 \leq x_1 + x_2 \\
         0 \leq & x_5 \leq 0.5x_3+1 \\ 
         0 \leq & x_6 \leq 0.5x_4 + 1 \\
         x_5 \leq & x_7 \leq x_5 \\ 
         x_6+1 \leq & x_8 \leq x_6+1 \\
         x_7 & > x_8 \text{ (negation of property)}
    \end{aligned}
\label{eq:conjunction}
\end{equation}

We find the set of important neurons (marked neurons), and repair only marked neurons. 
When \deeppoly{} fails to verify the network, we use an \milp{} solver to check the satisfiability
on equation~\ref{eq:conjunction}. 
If it return \unsat{} it means the property verified otherwise we get the satisfying assignment of each variables. 
The satisfying assignment of equation~\ref{eq:conjunction} are in equation~\ref{eq:sat1}
We execute the satisfying assignment $x_1=1,x_2=1$ on neural network and get the values on each neurons 
as shown in equation~\ref{eq:sat2}. 
The output values $x'_7=2, x'_8=3$ satisfying the property, so, the input $x_1=1, x_2=1$ is an spurious counter example. 
We have to remove spurious counter example by doing the refinement analysis. 
We are using two approaches to find the marked neurons guided by the spurious counter example, 
and two approaches to refine (repair) the marked neurons.

\begin{equation}
    \begin{aligned}
        x_1=1, x_2=1, x_3=2, x_4=2, x_5=2, x_6=0, x_7=2, x_8=1
    \end{aligned}
\label{eq:sat1}
\end{equation}

\begin{equation}
    \begin{aligned}
        x'_1=1, x'_2=1, x'_3=2, x'_4=2, x'_5=2, x'_6=2, x'_7=2, x'_8=3
    \end{aligned}
\label{eq:sat2}
\end{equation}




\begin{figure}[!ht]
	\centering
	\scalebox{0.8}{\begin{tikzpicture}[
    % define styles 
    clear/.style={ 
        draw=none,
        fill=none
    },
    net/.style={
        matrix of nodes,
        nodes={ draw, circle, inner sep=3pt },
        nodes in empty cells,
        column sep=1cm,
        row sep=.1cm
    },
    >=latex
]
% define matrix mat to hold nodes
% using net as default style for cells
\matrix[net] (mat)
{
% Define layer headings
   |[clear]| \parbox{1.5cm}{\centering $\langle x_1 \geq -1$, \\ $x_1 \leq 1$, \\ $l_1 = -1$, \\ $u_1 = 1 \rangle$} 
   & |[clear]| \parbox{2cm}{\centering $\langle x_3 \geq x_1 + x_2$, \\ $x_3 \leq x_1 + x_2$, \\ $l_3 = -2$ \\ $u_3 = 2 \rangle$}
   & |[clear]| \parbox{2.2cm}{\centering $\langle x_5 \geq 0$, \\ $x_5 \leq 0.5.x_3 + 1$, \\ $l_5 = 0$, \\ $u_5 = 2 \rangle$}
   & |[clear]|  \parbox{1.5cm}{\centering $\langle x_7 \geq x_5$, \\ $x_7 \leq x_5$, \\ $l_1 = 0$, \\ $u_1 = 2 \rangle$} \\
        
$x_1$  & $x_3$  & $x_5$ & $x_7$ \\
|[clear]| & |[clear]| & |[clear]| & |[clear]| \\
|[clear]| & |[clear]| & |[clear]| & |[clear]| \\
|[clear]| & |[clear]| & |[clear]| & |[clear]| $0.2$ \\
$x_2$  & $x_4$  & $x_6$ & $x_8$ \\
|[clear]| \parbox{1.5cm}{\centering $\langle x_2 \geq -1$, \\ $x_2 \leq 1$, \\ $l_2 = -1$, \\ $u_2 = 1 \rangle$} 
& |[clear]| \parbox{2cm}{\centering $\langle x_4 \geq x_1 + x_2$, \\ $x_4 \leq x_1 + x_2$, \\ $l_4 = -2$ \\ $u_4 = 2 \rangle$}
& |[clear]| \parbox{2.2cm}{\centering $\langle x_6 \geq 0$, \\ $x_6 \leq 0.5.x_4 + 1$, \\ $l_6 = 0$, \\ $u_6 = 2 \rangle$}
& |[clear]| \parbox{1.5cm}{\centering $\langle x_8 \geq x_6$, \\ $x_8 \leq x_6$, \\ $l_1 = 0$, \\ $u_1 = 2 \rangle$} \\
};

% left most lines into input layers
\draw[<-] (mat-2-1) -- +(-1cm,0) node[pos=1, above] {$[-1,1]$};
\draw[<-] (mat-6-1) -- +(-1cm,0) node[pos=1, above] {$[-1,1]$};

\draw[->] (mat-2-1) -- (mat-2-2) node[pos=0.5, above] {$1$};
\draw[->] (mat-2-1) -- (mat-6-2) node[pos=0.25, above] {$1$};
\draw[->] (mat-6-1) -- (mat-2-2) node[pos=0.25, above] {$1$};
\draw[->] (mat-6-1) -- (mat-6-2) node[pos=0.5, above] {$1$};

\draw[->] (mat-2-2) -- (mat-2-3) node[pos=0.5, above] {$max(0,x_3)$};
\draw[->] (mat-6-2) -- (mat-6-3) node[pos=0.5, above] {$max(0,x_4)$};

\draw[->] (mat-2-3) -- (mat-2-4) node[pos=0.5, above] {$1$};
\draw[->] (mat-2-3) -- (mat-6-4) node[pos=0.25, above] {$0$};
\draw[->] (mat-6-3) -- (mat-2-4) node[pos=0.25, above] {$0$};
\draw[->] (mat-6-3) -- (mat-6-4) node[pos=0.5, above] {$1$};


%\draw[->] (mat-2-4) -- +(1cm,0) node[pos=1, above] {$[0,3]$};
%\draw[->] (mat-6-4) -- +(1cm,0) node[pos=1, above] {$[-2,2]$};



% lines from a_{i}^{1} to a_{0}^{2}
%\foreach \ai in {1,2}
%  \draw[->] (mat-\ai-2) -- (mat-6-3);
   
% right most line with Output label
%\draw[->] (mat-6-3) -- node[above] {Output} +(2cm,0);
\end{tikzpicture}
}
	\caption{Hypothetical example of neural network}
	\label{fig:motivating}
\end{figure}


Following are the approaches to find the marked neurons.
\\
\textbf{Pullback approach}\\
% The core idea of this approach is to find a point in the predecessor layer, such that the current point 
% can be reached from the predecessor layers point. 
This approach finds a point of $x_7=2,x_8=1$ in the predecessor layer means the value of $x_5$ and $x_6$. 
The point $x_7=2$ and $x_8=1$ must reach from the values of $x_5$ and $x_6$. 
We compute the value by using the \sat{} query on the affine layer. The \sat{} query shown 
in equation~\ref{eq:back1}, 
build by using the values and affine expression of $x_7$, $x_8$, and by using the bounds of $x_5$ and $x_6$.
The satisfying value of equation~\ref{eq:back1} is $x_5=2$ and $x_6=0$, which is the predecessor point of $x_7=2$
and $x_8=1$.  
\begin{equation}
    \begin{aligned}
        x_7 = 2 & \land x_8 = 1 \\
        x_5 = 2 & \land x_6 + 1 = 1 \\ 
        x_5=2\land x_6+1 = 1 & \land 0\leq x_5 \leq 2 \land 0\leq x_6 \leq 2 \\
    \end{aligned}
\label{eq:back1}
\end{equation}

In this approach we compute the predecessor point by considering both \texttt{affine} and \texttt{relu} layer together. 
We compute the predecessor point of $x_5=2, x_6=0$ in terms of $x_1$ and $x_2$.
Since $x_6=0$ and $x_6=max(0,x_4)$ then $x_4$ can be anything from its lower bound to $0$ i.e. $-2 \leq x_4 \leq 0$.
Similarly the value of $x_5=2$ and $x_5=max(0,x_3)$ then the value of $x_3$ will be $2$. 
The \sat{} query build as shown in equation~\ref{eq:back2}. 

\begin{equation}
    \begin{aligned}
        x_5 = 2 & \land x_6 = 0 \\
        x_3 = 2 & \land -2\leq x_4 \leq 0 \\ 
        x_1+x_2=2\land -2\leq  x_1+x_2 \leq 0 & \land -1\leq x_1 \leq 1 \land -1\leq x_2 \leq 1 \\
    \end{aligned}
\label{eq:back2}
\end{equation}

When we check the satisfiability of equation~\ref{eq:back2}, it return \unsat{}. It means the predecessor point of 
$x_5=2,x_6=0$ does not exits, which implies that point $x_5=2, x_6=0$ is introduced by the triangle approximation.
Whenever the solver return \unsat{}, we also get the \unsatcore{}~\ref{sec:solver}. In this example we get the 
\unsatcore{}=$\{x_1+x_2=2, -2\leq  x_1+x_2 \leq 0\}$, so, we mark the corresponding neurons $x_3$ and $x_4$ as a marked neurons. 
And refine the marked neurons in the refinement procedure. 
\\
\textbf{Optimization based approach}\\
The satisfying assignments of equation~\ref{eq:conjunction} are in equation~\ref{eq:sat1}. 
We execute the satisfying assignment $x_1=1,x_2=1$ on neural network and get values on each neurons as 
shown in equation~\ref{eq:sat2}. We make the point on \texttt{relu} layer as close as possible 
to the point of \texttt{relu} in equation~\ref{eq:sat2}, such that input points $x_1,x_2$ and 
output point $x_7,x_8$ remain same as in equation~\ref{eq:sat1}. 
The equation~\ref{eq:opt1} shows the constrains which must be satisfied. The first and the last line of equation~\ref{eq:opt1}
ensure that the input and output points remain same as in equation~\ref{eq:sat1}. 
The constrains $\{x_5=2, x_6=2\}$ are soft constraints, which means maximum number of soft constraints should satisfy. 
The soft constraints define the closeness point on relu layer of equation~\ref{eq:sat1} and \ref{eq:sat2}. 
Finally, we give to the optimizer the constraints in equation~\ref{eq:opt1} and the soft constraints. 
The constraints of equation~\ref{eq:opt1} must satisfied because it is the same equation as of \ref{eq:conjunction}
except the fixed input and output points. 
The optimizer returns the constraints from soft constrains which are satisfied with the constraints 
of equation~\ref{eq:opt1}. If optimizer returns all soft constraints($\{x_5=2,x_6=2\}$) it means the output point
$x_7=2, x_8=1$ can be reached from $x_5=2,x_6=2$. But optimizer returns the soft constrains $\{x_5=2\}$, it mean 
$x_6$ is a neurons which is contributing to the output point $x_7=2, x_8$. We mark the neuron $x_6$ as 
marked neurons.  

\begin{equation}
    \begin{aligned}
        x_1 = 1 & \land x_2 = 1 \\
        x_3 = x_1 + x_2 & \land x_4 = x_1 + x_2 \\
        0 \leq x_5 = 0.5x_3 + 1 & \land 0\leq x_6 \leq 0.5x_4 + 1 \\ 
        x_7 = x_5 & \land x_8 = x_6 + 1 \\
        x_7 = 2 & \land x_8 = 1
    \end{aligned}
\label{eq:opt1}
\end{equation}
\\
\textbf{Refinement} \\
We have two approaches for the refinement {\em Splitting based approach} and {\em MILP based approach}.
We elaborate here the {\em MILP based approach} only. The other approach can be seen in algorithm~\ref{algo:verif2}. 
These two approaches refine the marked neurons. We got two different set of marked neurons in the above analysis. 
The {\em pullback} approach returns $\{x_5, x_6\}$ as the marked neurons, while {\em optimization based approach}
return the $\{x_6\}$ as the marked neurons. In the following analysis we are taking the marked neurons set as $\{x_5, x_6\}$.
Although the analysis will also work if we takes the other set of marked neurons.

In {\em MILP based approach}, we add the exact encoding of marked neuron ($x_5, x_6$) in addition to the constraints
in equation~\ref{eq:conjunction} and check the satisfiability, now it become \unsat{}, hence, the property verified. 
The exact constraint of a \relu{} neuron is explained in equation~\ref{eq:reluexact}. 

The {\em Splitting based approach} splits the bounds at $0$ of the incoming neurons of the marked neurons. 
The incoming neurons of $x_5$ and $x_6$ are $x_3$ and $x_4$ respectively. 
The affine expression of both the incoming neurons is $x_1+x_2$. 
When we split both $x_3$ and $x_4$ then four cases get generated. The property should get verified in all four cases. 
All the cases are explained as follow: 
% Suppose I split $x_3$ then two cases generated
% $x_3 > 0$ and $x_3 \leq 0$, which implies $x_1+x_2 > 0$ and $x_1+x_2 \leq 0$ respectively. 
% When we split both $x_3$ and $x_4$ then four cases get generated. Out of four cases $x_3 > 0 \land x_4 \leq 0$ and 
% $x_3 \leq 0 \land x_4 > 0$ are infeasible cases, because the corresponding affine expression 
% $x_1+x_2 > 0 \land x_1 + x_2 \leq 0$ is infeasible. We analyze the remaining tow cases as follow: 
\begin{itemize}
    \item $x_3 > 0$ and $x_4 > 0$, since neurons are in active phase, so, upper and lower expressions for $x_5$ 
    becomes $x_3$, similarly the the upper and lower expression for $x_6$ become $x_4$. 
    The expression for $x_7$ and $x_8$ remains the same. The upper bound of $x_7 - x_8$ computed in equation~\ref{eq:split1} is $-1$. 
    The property got verified.   
    \item $x_3 \leq 0$ and $x_4 \leq 0$, since both neurons are in passive phase, so, the upper and lower expressions 
        for both the neurons become $0$. The upper and lower expressions of the other neurons remain same. 
        The upper bound of $x_7 - x_8$ is computed $-1$ in equation~\ref{eq:split2}. The property got verified here too. 
    \item $x_3 \leq 0$ and $x_4 > 0$, is infeasible case because $x_1+x_2 \leq 0 \land x_1 + x_2 > 0$ is infeasible. 
    \item $x_3 > 0$ and $x_4 \leq 0$, is infeasible case because $x_1+x_2 > 0 \land x_1 + x_2 \leq 0$ is infeasible. 
\end{itemize}

\begin{equation}
    \begin{aligned}
        x_7 - x_8 \leq & x_5 - x_6 - 1 \\
       \leq & x_3 - x_4 -1 \\
       \leq & x_1 + x_2 -(x_1+x_2) -1 \\
       \leq & -1 
    \end{aligned}
    \label{eq:split1}
\end{equation}

\begin{equation}
    \begin{aligned}
      x_7 - x_8 \leq & x_5 - x_6 - 1 \\
        \leq & 0 - 0 - 1 \\
        \leq & -1
    \end{aligned}
    \label{eq:split2}
\end{equation}



% \begin{equation}
%     \begin{align}
%         \text{softConstrs} = \{x_5=2, x_6=0\}
%     \end{align}
% \end{equation}

