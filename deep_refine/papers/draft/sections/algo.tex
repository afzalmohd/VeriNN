In this section, we present our method to refine \deeppoly in a novel way.
%
\texttt{DeepPoly} is a sound and incomplete technique because it does over-approximation analysis. 
If \deeppoly{} verifies the property then the property is guaranteed to be verified, otherwise, the
result of it is unknown. 
We overcome this limitation by using a CEGAR-like technique, which is a complete technique
and rely on~\deeppoly{}. 
In our refinement approach, we mark some Relu neurons to have exact behavior on top
of~\deeppoly{} constraints, which is the strategy of refinement in the most complete
state-of-the-art techniques~\cite{alphabeta,etc}.
We add the encoding of the exact behavior to the~\deeppoly{} constraints.
%
We use MILP solver on the extended constraints to
check if the extra constraints rule out all spurious counter-examples.
%
The calls to MILP solvers is expensive,
therefore
we use spurious counter-examples to identify a small set of marked neurons.


In algorithm~\ref{algo:main}, we present the top-level flow of our approach.
The algorithm takes a verification query $\langle N,P,Q \rangle$ as input and returns
success if the verification is successful, otherwise
returns a counter-example to the query.
The algorithm uses supporting algorithms $\textsc{getMarkedNeurons}$ and
$\textsc{isVerified}$ to
get more marked neurons to refine and check the validity of verification query under
the marked neurons respectively.

The first line of the algorithm~\ref{algo:main} generates all the abstract constraints
by using \deeppoly{}.
For a node $n_{ij} \in A.neurons$,
the abstract constraints consists of the lower and upper constraints as well as the lower and upper bounds.
Let $A.lc_i = \Land_{j=1}^{|l_i|} A(n_{ij}).lexpr \leq x_{ij} \leq  A(n_{ij}).uexpr$, which is a 
conjunction of upper and lower constraints of each neuron of layer $l_i$ with respect to abstract constraint $A$.
The $lexpr$ and $uexpr$ for any neuron of a layer contain variables only from the previous layer's neurons, 
hence $A.lc_i$ contains the variables from layers $l_{i-1}$ and $l_i$. 


%
At line 2, we initialize the variable $marked$ to the empty set of neurons.
%
At the next line, we iterate in a while loop until either we verify the query or
find a counterexample.
%
At line 4, we call $\textsc{isVerified}$ with the verification query, abstraction $A$, and the
set of marked neurons.
%
In the verification step, the behavior of the marked neurons are encoded exactly, which
will be explained shortly.
%
The call either returns that the query is verified or returns an abstract counterexample,
which is defined as follows.

\begin{df}
  A sequence of value vectors $\boldsymbol{v_0}, \boldsymbol{v_1}, ... , \boldsymbol{v_k}$ is an 
  {\em abstract execution} of abstract constraint $A$ if 
  $\boldsymbol{v_0} \models lc_0$ and $\boldsymbol{v_{i-1}}, \boldsymbol{v_i} \models A.lc_i$ for each $i \in [1,k]$.  
  An abstract execution $\boldsymbol{v_0,...,v_k}$ is
  an {\em abstract counterexample} if~$\boldsymbol{v_k} \models \lnot Q$.
\end{df}


If these algorithms return verification successful then we report verified,
otherwise analyze the abstract counterexample $CEX(v_0,...,v_k)$.
At line 6, we first check if executing the neural network $N$ on input $v_0$
violates the predicate $Q$.
If yes, we report input $v_0$ for which the verification query is not true.
Otherwise, we the abstract counterexample is spurious.
We call $\textsc{getMarkedNeurons}$ to analyze the counterexample and
returns the causes of spuriousness, which is a set of neurons $markedNt$.
We add the new set $markedNt$ in the old set $marked$.
We iterate our loop again with the new set of marked neurons.

Now let us present $\textsc{isVerified}$ and $\textsc{getMarkedNeurons}$ in detail.

% Algorithm \ref{algo:verif1} exploits the markedNeurons to verify the property. 

%
% The algorithm uses supporting algorithms~\ref{algo:refine1} and~\ref{algo:verif1} to
% verify the query and return either the verification successful or abstractCEX. 

% The algorithm~\ref{algo:main} represents the high-level flow of our approach.



% Our approach contains two parts, the first part contains an approach to finding the markedNeurons. 
% The second part contains the verification approach (utilizing markedNeurons).


% In algorithm ~\ref{algo:refine2} and ~\ref{algo:verif1}
% They are presented in 

% \clearpage

% The goal is to find an input $\boldsymbol{x}$, such that the predicates $P(\boldsymbol{x})$ and 
% $Q(N(\boldsymbol{x}))$ holds. The predicate $Q$ usually is the negation of the desired property. 
% The triple  is our verification query.
%  \todo{How to formalize $P$ and $Q$}

% An input vector $\boldsymbol{v} \in \reals^{|l_0|}$ is a counter-example(cex) if $N(\boldsymbol{v}) \models \lnot Q$. 
% Let us say $satval_{ij}$ represents the satisfying value of variable $x_{ij}$, after an optimization query. 


% We will use the following convention in writing our algorithms.


% \begin{itemize}
% \item 
% \item 
% % \item $W_i, B_i$ represent the weight and bias matrix of layer $l_i$ respectively. 
% \end{itemize}


% \begin{df}
% \end{df}





\begin{algorithm}[t]
  \textbf{Input: } A verification problem $\langle N,P,Q \rangle$ \\
  \textbf{Output: } UNSAT or SAT
  \begin{algorithmic}[1]
    \State $A := deeppoly(N,P)$\Comment{deeppoly generate the abstract constraints}
    \State marked := \{\}
    \While{True}
      \State result = $\textsc{isVerified}$($\langle N,P,Q \rangle$ , A, marked)
      \If{result = CEX($\boldsymbol{v_0}, \boldsymbol{v_1} ... \boldsymbol{v_k}$)}
        \If{$N(\boldsymbol{v_0}) \models \lnot Q$}
          \State \textbf{return} Failed($\boldsymbol{v_0}$)
        \Else
        \State markedNt := $\textsc{getMarkedNeurons}$($\langle N,P,Q \rangle$ , A, marked, $\boldsymbol{v_0}, \boldsymbol{v_1} ... \boldsymbol{v_k}$)
          \State marked := marked $\union$ markedNt
        \EndIf
      \Else
        \State \textbf{return} verified
      \EndIf
    \EndWhile
  \end{algorithmic}
  \caption{A CEGAR based approach of neural network verification}
  \label{algo:main}
\end{algorithm}

\begin{algorithm}[t]
  \textbf{Name: } $\textsc{isVerified}$ \\
  \textbf{Input: } Verification query $\langle N,P,Q \rangle$, abstract constraints $A$, and marked $\subseteq ~ N.neurons$ \\
  \textbf{Output: } verified or an abstract counterexample. 
  \begin{algorithmic}[1]
    \State $constr := P \land (\Land_{i=1}^k A.lc_i)\land \neg Q$
    \State $constr := constr \land (\Land_{x \in markedNeurons} exactConstr(x))$ \Comment{as in eq \ref*{eq:reluexact}}
    \State isSat = checkSat(constr)
    \If{isSat}
      \State m := getModel(constr)
      \State \textbf{return} CEX($m(x_0),....,m(x_k)$)
    \Else
      \State \textbf{return} verified
    \EndIf
  \end{algorithmic}
  \caption{Verify $\langle N,P,Q \rangle$ with abstraction A}
  \label{algo:verif1}
\end{algorithm}

\begin{algorithm}[t]
  \textbf{Name: } $\textsc{getMarkedNeurons}$ \\
  \textbf{Input: } Neural network $N$, \deeppoly{} abstraction $A$, $marked\subseteq N.neurons$, and abstract counterexample $({v_0}, {v_1} ... {v_k})$\\
  \textbf{Output: } New marked neurons. 
  \begin{algorithmic}[1]
    % \State \textbf{return} $\boldsymbol{x}$ if $N(\boldsymbol{x}) \models \neg Q$. 
    \State Let ${val^{v_0}_{ij}}$ be the value of $n_{ij}$, when $\boldsymbol{v_0}$ is input of $N$. 
    \For{$i=1$ to $k$} \Comment{inputLayer excluded}
      \If{$l_i$ is $\relu${} layer}
        \State $constr := \Land_{t=i}^k A.lc_t$
        \State $constr := constr \land (\Land_{n \in marked} exactConstr(n))$ \Comment{as in eq \ref{eq:reluexact}}
        \State $constr := constr \land \Land_{j=1}^{|l_{i-1}|} (x_{(i-1)j} = val^{v_0}_{(i-1)j})$
        \State $constr := constr \land \Land_{j=1}^{|l_k|} (x_{kj} = v_{kj})$
        \State $softConstrs := \cup_{j=1}^{|l_j|} (x_{ij} = val^{v_0}_{ij})$
        \State $res, softsatSet = MaxSat(constr, softConstrs)$ \Comment{res always \texttt{SAT}}
        \State $newMarked := \{n_{ij} | 1 \leq j \leq |l_i| \land (x_{ij} = val_i(j)) \notin  softsatSet\}$ 
        \If{$newMarked$ is empty}
          \State \textbf{continue}
        \Else
          \State \textbf{return} newMarked
        \EndIf 
      \EndIf
    \EndFor
  \end{algorithmic}
  \caption{Marked neurons from counterexample}
  \label{algo:refine2}
\end{algorithm}


\subsection{Verifying query under marked neurons}

In algorithm~\ref{algo:verif1}, we present the implementation of 
$\textsc{isVerified}$, which takes the verification query, the \deeppoly{} abstraction $A$,
and a set of marked neurons as input.
At line 1, we construct constraints $contr$ that encodes the executions that satisfy abstraction
$A$ at every step.
At line 2, we also include constraints in $constr$ that encodes exact behavior of the marked
neurons. The following is the encoding of the exact behavior~\cite{exact} of neuron $n_{ij}$.
\begin{align}
    \label{eq:reluexact}
    \begin{split}
      exactConstr(n_{ij}) := x_{(i-1)j} \leq x_{ij} &\leq x_{(i-1)j} - A(n_{(i-1)j}).lb*(1-a) \;\;\land  \\
        0 \leq x_{ij} &\leq A(n_{(i-1)j}).ub*a \land a \in \{0,1\} \\ 
    \end{split}
\end{align}
where $a$ is a fresh variable for each neuron.

At line 3, we call a solver to find a satisfying assignment of the constraints.
If $constr$ is satisfiable, we get a model $m$.
From the model $m$, we extract an abstract counterexample and return it.
If $constr$ is unsatisfiable, we return that the query is verified.


% \subsection{Causes of spuriousness}

% We have a following approach to find the causes of spuriousness (markedNeurons). 

\subsection{Maxsat based approach to find the marked neurons}

In algorithm~\ref{algo:refine2}, we present $\textsc{getMarkedNeurons}$
which analyzes an abstract spurious counterexample.
In our abstract constraints, we encode $\affine${} neurons exactly, but
over-approximate $\relu${} neurons.
We identify a set of
marked neurons whose exact encoding will eliminate the counterexample
in the future analysis.
%

Let ${val_i^{\boldsymbol{v_0}}}$ represent the value vector on layer $l_i$, 
if we execute the neural network on input $\boldsymbol{v_0}$.
%
\todo{Redefining ${val_i^{\boldsymbol{v_0}}}$ either remove here or in prelim}
Let us say ${v_0}, {v_1}, ... ,{v_k}$ is an abstract spurious counterexample.
We iterative modify the counterexample such that it becomes more and more
like ${val_i^{\boldsymbol{v_0}}}$.
Initially, ${val_0^{\boldsymbol{v_0}}}$ is equal to $v_0$.
Since we encode the affine layer exactly in $A.lc_i$, the following
theorem is trivially true.

% % 
% $\boldsymbol{v_0}$ is a counterexample if $\boldsymbol{val_k^{\boldsymbol{v_0}}}$ 
% does not satisfy the property $Q$.

\begin{theorem}
  \label{th:marked2}
  Let us say ${v_0}, {v_1}, ... {v_k}$ is an abstract execution. 
  If $Type_i = \affine$, ${val_i^{{v_0}}} = {v_i}$ if ${val_{i-1}^{{v_0}}} = {v_{i-1}}$.  
\end{theorem}
% \begin{proof}
%   Since affine layer's constraints are exact.
% \end{proof}
% Since $l_0$ is an input layer, so, ${v_0}$ and ${val_0^{v_0}}$ are equal,

By the above theorem, ${v_1}$ and ${val_1^{v_1}}$ are also equal. 
The core idea of our algorithm is to find ${v'_2}$ as close as possible to ${val_2^{v_0}}$, 
such that ${v_0}, {v_1}, {v'_2}, ... {v'_{k-1}}, {v_k}$ becomes an abstract spurious counterexample. 
We measure closeness by the number of elements of ${v'_2}$ are equal to the 
corresponding element of vector ${val_2^{v_0}}$.
\begin{enumerate}
\item If ${v'_2}$ is equal to ${val_2^{v_0}}$ then ${v'_3}$ will also become 
  equal to ${val_3^{v_0}}$ due to theorem \ref{th:marked2}. 
  Now we move on to the next $\relu${} layer $l_4$ and try to find the similar point ${v'_4}$, such that 
  ${v_0}, {v_1}, {v_2}, {v_3},{v''_4}... {v''_{k-1}},{v_k}$ is an . 
  We repeat this process until the following case occurs. 
\item If at some $i$, we can not make ${v'_i}$ equal to ${val_i^{v_0}}$ then we collect the neurons
  whose values are different in ${v'_i}$ and ${val_i^{v_0}}$. We call them marked neurons.
\end{enumerate}

In the algorithm, the above description is implemented using maxsat solver.
The loop at line 2 iterates over $\relu${}~layers.
In lines 4-7, we build constraints that encode that from layer $i$ onward
\deeppoly~abstract constraints are satisfied,
the marked neurons have exact behavior, layer $(i-1)$ neurons have value equal to
${val_{i-1}^{v_0}}$, and the execution finishes at $v_k$.
At line 8, we construct soft constraints $softConstrs$, which encodes $x_{ij}$ is equal to $val_{i}^{v_0}$.
At line 9, we call maxsat solver. 
This call to maxsat solver will always find a satisfying assignment because
our hard constraints are always satisfiable.
The solver will also return a subset $softsatSet$ of the soft constraints.
At line 10, we check which soft constraints are missing in $softsatSet$.
The corresponding neurons are added in $newMarked$.
If $newMarked$ is empty, we have managed to find a spurious
abstract counterexample from $val^{v_0}_{i}$.
We need to go to the next layer.
Otherwise, we return the new set of marked neurons.

% \subsection{Utilizing of spurious information}
% The \emph{isVerified} function in algorithm~\ref{algo:main} calls algorithm~\ref{algo:verif1}. 
% The algorithm~\ref{algo:verif1} takes \markednewrons{} as input. 
% The \markednewrons{} represent the set of the culprit neurons. 
% Algorithm~\ref{algo:verif1} replaces the abstract constraints of \markednewrons{} by exact constraints and check for the 
% property by \milp{} solver. If \milp{} solver return \sat{} then we return satisfying assignments as an abstractCEX, 
% otherwise, return verified. 
















%--------------------- DO NOT ERASE BELOW THIS LINE --------------------------

%%% Local Variables:
%%% mode: latex
%%% TeX-master: "../main"
%%% End:

