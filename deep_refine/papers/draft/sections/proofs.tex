\subsection{Proofs of progress and termination}

Our refinement strategy ensures progress, i.e., the spurious counterexample does not
repeat in the future
iterations of Algorithm~\ref{algo:main}. Let us suppose the algorithm
$\textsc{getMarkedNeurons}$ get abstract spurious counterexample
${v_0}, {v_1}, ... {v_k}$ and
returns marked neurons in $i$th iteration.
The call to $\maxsat{}$ at line 10
declares that $constr \land softsatSet$ is satisfiable.
We can extract an abstract spurious counterexample from a model
of $constr \land softsatSet$.
Let $m$ be the model.
Let the abstract spurious counterexample be $cex = val_0^{v_0}, ...., val^{v_0}_{i-1},v'_{i},...,v'_{k-1},v_k$.
Before the step $i$, $cex$ follows the execution of $N$ on input $v_0$.
After the step $i$, we use the model to construct $cex$, i.e., $m(x_i) = v'_i$.

\begin{theorem}
  In the rest of the run of Algorithm~\ref{algo:main}, we will never find the abstract
  spurious counterexample
  $cex$ again.
\end{theorem}
\begin{proof}
  For $n_{ij} \in newMarked$, the \maxsat{} query ensures that
  $val^{v_0}_{ij} \neq v'_{ij}$.
  If we have the same counterexample again in the future then
  input of $n_{ij}$ will be $val^{v_0}_{(i-1)j}$.
  Since we will have exact encoding for $n_{ij}$, the output
  will be $val^{v_0}_{ij}$, which contradicts the earlier inequality.  
\end{proof}

Furthermore, we also prove that our method always terminates.
%
Theorems \ref{th:progress1} and \ref{th:progress2} imply that in every
iteration $\textsc{getMarkedNeurons}$
returns a non empty set of unmarked neurons, which we will now be marked. In worst case the algorithms marked all 
the neurons of the network, and encode them in the exact behavior. 

% Suppose the algorithm {\em getMarkedNeurons} get abstractCEX 
% ${v_0}, {v_1}, ... {v_k}$ as input, 
% and ${val_i^{{v_0}}}$ represents the value vector on layer $l_i$, 
% when we execute the neural network on input ${v_0}$.




\begin{theorem}
  \label{th:progress1}
  In every refinement iteration $\textsc{getMarkedNeurons}$
  returns a non-empty set of marked neurons. 
\end{theorem}

\begin{proof}
By the definition of abstract spurious counterexample, ${v_k} \models \lnot Q$. 
By the check at line 6 of Algorithm~\ref{algo:main} ${val_k^{{v_0}}} \models Q$. 
If the set of returned new marked neurons is empty, $newMarked = \emptyset$ for each layer. 
Therefore, all the neurons in any layer $l_i$ become equal to ${val_i^{{v_0}}}$,  
which implies ${v_k}$ equals to ${val_k^{{v_0}}}$, but ${v_k} \models \lnot Q$ and 
${val_k^{{v_0}}} \models Q$, which is a contradiction.   
\end{proof}
% {\em Proof by contradiction:} 


\begin{theorem}
  \label{th:progress2}
  In every refinement iteration $\textsc{getMarkedNeurons}$
  returns the marked neurons, which was not marked in the previous iterations. 
\end{theorem}

\begin{proof}
We will show, if a neurons $n_{ij}$ got marked in $t^{th}$ iteration then $n_{ij}$ will not marked again 
in any iteration greater than $t$.
Consider an iteration $t' > t$, if we get the marked neurons from layer other
than $l_i$ then $n_{ij}$ can not be part of it because $n_{ij}$ is in layer $l_i$. 
Consider the case where marked neurons are from the layer $l_i$ in $t'^{th}$ iteration. 
Since we have made $n_{ij}$ exact in line 6 of Algorithm~\ref{algo:refine2},
its behavior in constraints optimization will be same as the exact \relu{}. 
Moreover, $v'_{ij} = val_{ij}^{v_0}$, 
which implies the soft constraint for neuron $n_{ij}$ will always be satisfied. Hence it will not occurs in 
marked neurons as per the criteria of marked neurons in line 11 of Algorithm~\ref{algo:refine2}.   
\end{proof}


%%% Local Variables:
%%% mode: latex
%%% TeX-master: "../main"
%%% End:
