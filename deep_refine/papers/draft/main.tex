\documentclass[10pt]{llncs}


\input{macro.tex}
\setlength{\textfloatsep}{1mm}% Remove \textfloatsep

\title{A CEGAR Based Verification of Neural Network}

\author{Mohammad Afzal\inst{1,2}\and Ashutosh Gupta\inst{1}\and Akshay S\inst{1}}

\institute{Indian Institute of Technology, Bombay, India\and TCS Research, Pune, India}
% \date{April 2021}

\begin{document}

\maketitle

\begin{abstract}
       Neural networks have achieved an important role in safety-
       critical systems, e. g., autonomous vehicles, medical diagnosis, and speech
       recognition. However, even the state of the arts network can easily be
       fooled by carefully making small modifications in input. The researchers
       found a need here to formally verify the desired properties of the neural net-
       works. The neural network verification problem is an NP-hard problem.
       Therefore, practical methods need to resolve the trade-off between scalability and precision. The state of the arts, which scales on the real-world
       networks are based on the abstraction-based approach. These tools suffer
       from imprecision. We are trying to remove imprecision by refining the
       abstraction. The tool on which our method work is \deeppoly{}. Our experiments show that we are outperforming in comparison to the related
       refinement techniques, also verifying the reasonable number of unique
       benchmarks on which the state of the arts fails to verify. 
\end{abstract}

\section{Introduction}
\label{sec:intro}
%% one paragraph per point
% about 2 pages 



% Current state

% Zoom into deeppoly

% your technique (2 para )


% Experiments:
% tools (why those tools), benchmark, results

% Layout of the rest of the paper


% Why this problem?
In the recent years, neural networks are being increasingly used in
safety-critical systems such as autonomous vehicles,
medical diagnosis, and speech recognition~\cite{bojarski2016end,amato2013artificial,hinton2012deep}. Since the neural networks are trained using data, they may not give the expected output for each unseen input. Goodfellow \cite{goodfellow2014explaining} have shown that a slight change in the input can fool the neural networks, i.e., image misclassification by the change in a few pixels.  The developers find it hard to analyze/debug the neural networks because they contain hundreds of thousands of non-linear nodes.

% What is the problem?

To address this problem, developers need automatic verification of the networks.
Since automatic verification of the neural network is an NP-hard problem, therefore researchers use approximations in their methods. We may divide the methods into two classes, namely complete and incomplete. The methods~\cite{lomuscio2017approach,fischetti2018deep,dutta2018output,cheng2017maximum,katz2017reluplex,katz2019marabou,ehlers2017formal,huang2017safety,wang2021beta,xu2020fast,zhang2022general} are complete methods, i.e., they explore the exact state space.
Since complete methods explore exact state space, they suffer from scalability issues on the large-scale networks. On the other hand, the abstraction based methods \cite{dvijotham2018dual}, \cite{gehr2018ai2}, \cite{singh2018fast},
 \cite{singh2018boosting}, \cite{weng2018towards}, \cite{wong2018provable}, \cite{zhang2018efficient}, \cite{zhang2018efficient} are sound and incomplete because they over approximate the state space. The incomplete methods scale well in comparison to complete methods.  The abstraction based methods suffer from imprecision, so, the methods \cite{wang2018formal,wang2018efficient,elboher2020abstraction,yang2021improving,lin2020art} refine the over-approximated state space to achieve completeness. In \cite{wang2018formal,wang2018efficient,lin2020art} the authors eliminate the spurious information by bisecting the input space on the guided dimension. The \cite{yang2021improving} works on top of \deeppoly{}~\cite{singh2019abstract}, they remove
the spurious region by taking each neuron's constraints with the negation of the property and using the  Gurobi optimizer \cite{gurobioptimizer} to refine the bounds of neurons. Elbohar et.al. \cite{elboher2020abstraction} defined the four classes of neurons based on their characteristics. At the time of abstraction, they merge each neuron into one of the four classes.  After completing the abstraction process they use the state of the complete verifier to verify  the abstract network, they go to the refinement process in case of failure.   In the refinement, process authors split the merged neurons.  In the worst case, this method may get back to the original network.  Although this work is a cegar-based approach, it also suffers from scalability issues on large-scale networks.   The work \cite{elboher2020abstraction} refinement process focuses on the structure of the networks.

Our method is top on the \deeppoly{}.  \deeppoly{} maintains a single upper and a single lower linear constraints as well as lower and upper bounds
for each neuron in the network. For an affine neuron, the upper and lower constraints are the same as an affine expression, 
which is a weighted sum of the input neurons. For a \relu{} neuron, they are constructing the upper and lower constraints
by doing the triangle approximation~\cite{singh2019abstract}. 
Triangle approximation is also explained in section~\ref{sec:deeppoly}. 
\deeppoly{} scales well on large-scale networks, but it also suffers from imprecision. 

In this paper, we remove the imprecision by doing a novel refinement, which is cegar-based. 
Whenever \deeppoly{} fails to verify the property, we collect all the linear constraints generated by it, 
with the negation of the property, and check for satisfiability, by using an \milp{}-based tool. 
If the tool return \unsat{} then we report property \texttt{verified}. Otherwise, we go to the refinement process.
We have two parts of our refinement approach, one finds the causes of spurious information 
and the second part refines the information gets in the first part. 
We have two methods to find the causes of spuriousness namely the {\em pullback} method, and the 
{\em optimization-based} method. And two methods for the refinement namely {\em splitting} based method and 
{\em MILP based}-method.    

We explain our approach with a motivating example in the next section~\ref{sec:motivation}. 
We define the notions and definitions in section~\ref{sec:model}. The section~\ref{sec:algo} contains the 
algorithm procedure of our approach and section~\ref{sec:experiments} contains the experiments. 
We present the related work and future work in sections \ref{sec:related} and \ref{sec:conclusion} respectively. 


% --- do not erase below this line ----

%%% Local Variables:
%%% mode: latex
%%% TeX-master: "../main"
%%% End:


% \clearpage

\section{A Motivating Example}
\label{sec:motivation}
\begin{figure}[t]
	\centering
	\scalebox{0.8}{\begin{tikzpicture}[
    % define styles 
    clear/.style={ 
        draw=none,
        fill=none
    },
    net/.style={
        matrix of nodes,
        nodes={ draw, circle, inner sep=3pt },
        nodes in empty cells,
        column sep=1cm,
        row sep=.1cm
    },
    >=latex
]
% define matrix mat to hold nodes
% using net as default style for cells
\matrix[net] (mat)
{
% Define layer headings
   |[clear]| \parbox{1.5cm}{\centering $\langle x_1 \geq -1$, \\ $x_1 \leq 1$, \\ $l_1 = -1$, \\ $u_1 = 1 \rangle$} 
   & |[clear]| \parbox{2cm}{\centering $\langle x_3 \geq x_1 + x_2$, \\ $x_3 \leq x_1 + x_2$, \\ $l_3 = -2$ \\ $u_3 = 2 \rangle$}
   & |[clear]| \parbox{2.2cm}{\centering $\langle x_5 \geq 0$, \\ $x_5 \leq 0.5.x_3 + 1$, \\ $l_5 = 0$, \\ $u_5 = 2 \rangle$}
   & |[clear]|  \parbox{1.5cm}{\centering $\langle x_7 \geq x_5$, \\ $x_7 \leq x_5$, \\ $l_1 = 0$, \\ $u_1 = 2 \rangle$} \\
        
$x_1$  & $x_3$  & $x_5$ & $x_7$ \\
|[clear]| & |[clear]| & |[clear]| & |[clear]| \\
|[clear]| & |[clear]| & |[clear]| & |[clear]| \\
|[clear]| & |[clear]| & |[clear]| & |[clear]| $0.2$ \\
$x_2$  & $x_4$  & $x_6$ & $x_8$ \\
|[clear]| \parbox{1.5cm}{\centering $\langle x_2 \geq -1$, \\ $x_2 \leq 1$, \\ $l_2 = -1$, \\ $u_2 = 1 \rangle$} 
& |[clear]| \parbox{2cm}{\centering $\langle x_4 \geq x_1 + x_2$, \\ $x_4 \leq x_1 + x_2$, \\ $l_4 = -2$ \\ $u_4 = 2 \rangle$}
& |[clear]| \parbox{2.2cm}{\centering $\langle x_6 \geq 0$, \\ $x_6 \leq 0.5.x_4 + 1$, \\ $l_6 = 0$, \\ $u_6 = 2 \rangle$}
& |[clear]| \parbox{1.5cm}{\centering $\langle x_8 \geq x_6$, \\ $x_8 \leq x_6$, \\ $l_1 = 0$, \\ $u_1 = 2 \rangle$} \\
};

% left most lines into input layers
\draw[<-] (mat-2-1) -- +(-1cm,0) node[pos=1, above] {$[-1,1]$};
\draw[<-] (mat-6-1) -- +(-1cm,0) node[pos=1, above] {$[-1,1]$};

\draw[->] (mat-2-1) -- (mat-2-2) node[pos=0.5, above] {$1$};
\draw[->] (mat-2-1) -- (mat-6-2) node[pos=0.25, above] {$1$};
\draw[->] (mat-6-1) -- (mat-2-2) node[pos=0.25, above] {$1$};
\draw[->] (mat-6-1) -- (mat-6-2) node[pos=0.5, above] {$1$};

\draw[->] (mat-2-2) -- (mat-2-3) node[pos=0.5, above] {$max(0,x_3)$};
\draw[->] (mat-6-2) -- (mat-6-3) node[pos=0.5, above] {$max(0,x_4)$};

\draw[->] (mat-2-3) -- (mat-2-4) node[pos=0.5, above] {$1$};
\draw[->] (mat-2-3) -- (mat-6-4) node[pos=0.25, above] {$0$};
\draw[->] (mat-6-3) -- (mat-2-4) node[pos=0.25, above] {$0$};
\draw[->] (mat-6-3) -- (mat-6-4) node[pos=0.5, above] {$1$};


%\draw[->] (mat-2-4) -- +(1cm,0) node[pos=1, above] {$[0,3]$};
%\draw[->] (mat-6-4) -- +(1cm,0) node[pos=1, above] {$[-2,2]$};



% lines from a_{i}^{1} to a_{0}^{2}
%\foreach \ai in {1,2}
%  \draw[->] (mat-\ai-2) -- (mat-6-3);
   
% right most line with Output label
%\draw[->] (mat-6-3) -- node[above] {Output} +(2cm,0);
\end{tikzpicture}
}
	\caption{Hypothetical example of neural network}
	\label{fig:motivating}
\end{figure}
Consider a neural network in figure~\ref{fig:motivating},  which has one input, one hidden, and one output layer. The hidden layer is separated into two layers 
\affine{} and \relu{} layers, so a total of four layers is shown in figure~\ref{fig:motivating}. 
Every layer contains two neurons. The neuron $x_8$ has a bias of $1$ and all the other neurons have a bias of $0$. 
Our goal is to verify for all input $x_1,x_2 \in [-1,1]$ the outputs satisfy $x_7 \leq x_8$. 
Our approach extends \deeppoly{}~\cite{singh2019abstract}. \deeppoly{} maintains the one upper and one lower constraint
as well as upper and lower bound for each neuron. For a neuron of the affine layer, the upper and lower constraint is 
the same, which is the weighted sum of the input neurons i.e. $x_3$'s upper and lower constraint is $x_1+x_2$.
For an activation neuron, the upper and lower expression is computed using triangle approximation~\cite{singh2019abstract}, 
which is briefly explained in section~\ref{sec:deeppoly}. To verify the property $x_7 \leq x_8$, \deeppoly{} creates a 
new expression $x_9 = x_7 - x_8$ and computes the upper bound of $x_9$. The upper bound of $x_9$ should not be greater
than $0$. \deeppoly{} computes the upper bound of $x_9$ by back substituting the expression of $x_7$ and $x_8$ 
from the previous layer.%, the back substitute further from the previous layer.
They continue back substituting until only input layer variables are left.
The process of back substitution is shown in equation (1). %~\ref{eq:deeppoly}.
After back substitution, the upper bound of $x_9$ is
computed as $1$, which is greater than $0$, 
hence, the \deeppoly{} fails to verify the property.

\hspace*{-1cm}
\fbox{
\noindent\begin{minipage}{.23\linewidth}
\begin{equation*}
    \begin{aligned}
        x_9 \leq  &  x_7 - x_8 &\\
        x_9 \leq  & x_5 - x_6 - 1 \\
        x_9 \leq  & 0.5x_3 + 1 - 1 &\\
        x_9 \leq  & 0.5(x_1+x_2) \\
        x_9 \leq  & 1\\
        (1)
   \end{aligned}
%\label{eq:deeppoly}
\end{equation*}
\end{minipage}}\quad
\fbox{
\begin{minipage}{.68\linewidth}
\begin{equation*}
    \begin{aligned}
      -1 \leq & x_1 \leq 1 \hspace*{-1cm}&  \hspace*{-2cm}
      -1 \leq & x_2 \leq 1 \\
         x_1 + x_2 \leq & x_3 \leq x_1 + x_2 \hspace*{-1cm}& \hspace*{-2cm}
         x_1 + x_2 \leq & x_4 \leq x_1 + x_2 \\
         0 \leq & x_5 \leq 0.5x_3+1 \hspace*{-1cm}&  \hspace*{-2cm}
         0 \leq & x_6 \leq 0.5x_4 + 1 \\
         x_5 \leq & x_7 \leq x_5 \hspace*{-1cm}&\hspace*{-2cm}
         x_6+1 \leq & x_8 \leq x_6+1 \\
         x_7 & > x_8 \text{ (negation of property)}&&\\
        & \hspace*{2cm}(2) &&
    \end{aligned}
%\label{eq:conjunction}
\end{equation*}

\end{minipage}
}  

There are two main reasons for the failure of \deeppoly{}. First, it cannot maintain the complete correlation 
between the neurons. In this example neurons $x_3$ and $x_4$ has the same expression $x_1+x_2$, so, they always
get the same value. But in the \deeppoly{} analysis process, it may fail to get the same value. Second, it uses triangle
approximation on \relu{} neurons.
We take the conjunction of upper and lower expressions of each neuron with the negation of the property
as shown in equation (2)%~\ref{eq:conjunction},
 and use the \milp{} solver to check satisfiability, thus addressing the first issue.  
The second issue can be resolved either by splitting the bound at zero of the 
affine node or by using the exact encoding (eq~\ref{eq:reluexact}) 
instead of triangle approximation. 
But both solutions increase the problem size exponentially in terms of \relu{} neurons and this results in a huge 
blowup if we repair every neuron of the network. 

So, the main hurdle towards efficiency is to find the set of important neurons (we call these {\em marked neurons}), 
and only repair these.  For this we crucially use the satisfying assignment obtained from the \milp{} solver.
%When \deeppoly{} fails to verify the network, we use an \milp{} solver to check the satisfiability of equation~\ref{eq:conjunction}. 
%If it returns \unsat{} it means the property is verified otherwise we get the satisfying assignment of each variable. 
For instance, a possible satisfying assignment of equation~(2)%\ref{eq:conjunction}
is in equation~\ref{eq:sat1}. We execute the neural network with the inputs $x_1=1,x_2=1$ and get the values 
on each neuron as shown in equation~\ref{eq:sat2}. 
Then we observe that the output values $x'_7=2, x'_8=3$ satisfy the property, 
so, the input $x_1=1, x_2=1$ is a spurious counterexample. 
The question is to identify the neuron whose abstraction lead to this imprecision.
%\vspace{-20mm}
\setcounter{equation}{2}
\begin{align}
  x_1=1, x_2=1, x_3=2, x_4=2, x_5=2, x_6=0, x_7=2, x_8=1 \label{eq:sat1} \\
  x'_1=1, x'_2=1, x'_3=2, x'_4=2, x'_5=2, x'_6=2, x'_7=2, x'_8=3 \label{eq:sat2}
\end{align}

%We have to remove spurious counter example by doing the refinement analysis. 
%We are using one approach to find the marked neurons guided by the spurious counter example,  and one approach to refine (repair) the marked neurons.

% We are using two approaches to find the marked neurons guided by the spurious counter example, 
% and two approaches to refine (repair) the marked neurons.
% \begin{equation}
%     \begin{aligned}
%         x_1=1, x_2=1, x_3=2, x_4=2, x_5=2, x_6=0, x_7=2, x_8=1\\
%     \end{aligned}
%         \label{eq:sat1}
% \end{equation}

% \begin{equation}
%     \begin{aligned}
%         x'_1=1, x'_2=1, x'_3=2, x'_4=2, x'_5=2, x'_6=2, x'_7=2, x'_8=3
%     \end{aligned}
% \label{eq:sat2}
% \end{equation}


%Following is the approach to find the marked neurons.
%\\
\noindent\textbf{Maxsat based approach to identify marked neurons}\\
%The satisfying assignments of equation~\ref{eq:conjunction} are in equation~\ref{eq:sat1}. We execute the satisfying assignment $x_1=1,x_2=1$ on the neural network and get values on each neuron as  shown in equation~\ref{eq:sat2}.
% We make the point on \texttt{relu} layer in eq~\ref{eq:sat1} as close as possible to the point 
% of \texttt{relu} in equation~\ref{eq:sat2} by encoding them as soft constraints (i.e.,  $\{x_5=2, x_6=2\}$) 
% while maintaining that the rest of the hard constraints are satisfied (see Equation~\ref{eq:opt1})
% e.g., input points $x_1,x_2$ and output points $x_7,x_8$ are the same etc. 
% %The soft constraints define the closeness point on the relu layer of equations~\ref{eq:sat1} and \ref{eq:sat2}. 
% Then we feed this to a maxsat solver \todo{check! afzal} to maximize the soft constraints, 
% while maintaining the hard constraints.
%   %the constraints in equation~\ref{eq:opt1} and the soft constraints. 
% %The constraints of equation~\ref{eq:opt1} must be satisfied because it is the same equation as \ref{eq:conjunction} except for the fixed input and output points. The optimizer returns the constraints from soft constraints which are satisfied with the constraints of equation~\ref{eq:opt1}.
%   If the solver returns all soft constraints($\{x_5=2,x_6=2\}$) it means the output point $x_7=2, x_8=1$ 
%   can be reached from $x_5=2,x_6=2$. But in this case the solver could return only the soft constraint 
%   $\{x_5=2\}$, which implies that $x_6$ is a neuron which is contributing to the output point $x_7=2, x_8=1$. 
%   We mark the neuron $x_6$ as a marked neuron. \todo{this should be written better. maybe the schematic picture should be here for THIS example not general.}
To find the neurons whose abstraction leads to imprecision, let us see figure~\ref{fig:pictorial1}. 
Here, the black line represents the spurious counterexample denoted by equation~\ref{eq:sat1}. 
The green line represents the exact execution of the input point of spurious counterexample 
which is denoted by equation~\ref{eq:sat2}. 
Our goal is to make the black line as close as possible to the green line from the first layer to the last layer, but the 
first and last points should remain the same i.e., $x_1=1,x_2=1,$ and $x_7=2, x_8=1$, the closest line is the blue line.
Here the closeness measures in terms of the equality of the neurons. 
The green and black points are the same for the input layer, i.e., $[1,1]$. On the first affine layer, $l_1$ 
also, the black point $v_1$ is the same as the green point $v'_1$ since the affine layer does not introduce any spurious information. 
For $l_2$, we try to make $v''_2$ close to $v'_2$, such that $v''_2$ reaches to the $v_3$. We do that by encoding 
them as soft constraints (i.e.,  $\{x_5=2, x_6=2\}$) 
while maintaining that the rest of the hard constraints are satisfied (see Equation~\ref{eq:opt1})
e.g., input points $v_0=v''_0$ and output points $v_3=v''_3$ remain same. 
We mark the neurons of the layer where the blue line starts diverging from the green line, i.e., $l_2$. 
The divergence we find by the \maxsat{} query. If \maxsat{} returns all the soft constraints as satisfied, it means
the blue point becomes equal to the green point. If \maxsat{} returns partial soft constraints as satisfied, 
we mark the neurons whose soft constraints are not satisfied. In our example, \maxsat{} returns 
soft constraints $\{x_5=2\}$, soft constraint of $x_6$ could not satisfied, so, we mark $x_6$.
The blue and black lines are the same for our motivating example since it contains only one \relu{} layer. 
However, in general, it may or may not same. We are finding the blue line (the closest to the green line) to mark the 
less number of marked neurons. 




\noindent  
\fbox{
    \begin{minipage}{0.5\linewidth}
\begin{equation}
    \begin{aligned}
        x_1 = 1 & \land x_2 = 1 \\
        x_3 = x_1 + x_2 & \land x_4 = x_1 + x_2 \\
        0 \leq x_5 = 0.5x_3 + 1 & \land 0\leq x_6 \leq 0.5x_4 + 1 \\ 
        x_7 = x_5 & \land x_8 = x_6 + 1 \\
        x_7 = 2 & \land x_8 = 1
    \end{aligned}
    \label{eq:opt1}
\end{equation}
\end{minipage}
  }\;\;
  \begin{minipage}{0.42\linewidth}
%\textbf{Refinement} \\
%We have an approach for the refinement named as {\em MILP-based approach}.
Once we have  $x_6$ as the marked neuron, we use an {\em MILP based approach}, and add the exact encoding of the marked neuron ($x_6$) in addition to the constraints in equation (2) %~\ref{eq:conjunction}
and check the satisfiability, now it becomes \unsat{}, hence, the property verified (see Eq. equation~\ref{eq:reluexact} for more details).
%The exact constraint of a $\relu${} neuron is explained in 
\end{minipage}

\begin{figure}[t]
    \centering
    \includegraphics[scale=0.75]{fig/pictorial1.pdf}
    \caption{Pictorial representation of our approach on example in figure~\ref{fig:motivating}}
    \label{fig:pictorial1}
\end{figure}


% \begin{figure}[!ht]
% 	\centering
% 	\scalebox{0.55}{\input{fig/pictorial1.tex}}
% 	\caption{Pictorial representation of our approach}
% 	\label{fig:pictorial}
% \end{figure}





%% \begin{figure}
%%     \begin{minipage}{0.9\textwidth}
%%         \centering
%%         \includegraphics[scale=0.8]{fig/pictorial1.pdf}
        
%%         (a)
%%     \end{minipage}

%%     \begin{minipage}{0.9\textwidth}
%%         \centering
%%         \includegraphics[scale=0.8]{fig/pictorial2.pdf}

%%         (b)
%%     \end{minipage}
%%     \caption{Pictorial representation of our approach}
%%     \label{fig:pictorial}
%% \end{figure}


%% Consider the pictorial representation of our approach in figure~\ref{fig:pictorial}. This pictorial representation is not 
%% related to the motivating example of figure~\ref{fig:motivating}. The shapes $p_0, p_1, p_2$, and $p_4$ represent 
%% the abstract constraints on each layer. We took the rectangle shape for simplicity,  but it is not necessarily a rectangle shape. 
%% The red zone on the output layer shows the intersection of the abstract constraints and the negation of the property. If this 
%% intersection is empty, we can say that the property was verified. Otherwise, we will get a path 
%% $v_0\rightarrow v_1\rightarrow v_2\rightarrow v_3$ as shown in figure~\ref{fig:pictorial}(a), which may or may not be a counterexample. 
%% We execute $v_0$ on neural network
%% and get path $v_0\rightarrow v'_1\rightarrow v'_2\rightarrow v'_3$ as shown in figure~\ref{fig:pictorial}(a). 
%% If the point $v'_3$ reaches the red zone, 
%% we report $v_0$ as a counterexample. Otherwise, we refine the path $v_0\rightarrow v_1\rightarrow v_2\rightarrow v_3$. 
%% To refine, we make the black line close to the green line from the first layer to the last layer so that the endpoints($v_0, v_3$)
%% remain unchanged. Here the closeness means making the neuron's values equal. 
%% First, we make the points $v_1$ close to $v'_1$, and it become completely equal (all neuron's values of $v_1$ become equal to the 
%% corresponding neuron's values of $v'_1$), we move on the the layer $l_2$. 
%% On layer $l_2$,
%% we make the points $v_2$ close to $v'_2$, and some neurons of $v_2$ become equal to the corresponding neurons of $v'_2$, and some
%% neurons could not become equal, so we got a new point $v''_2$ close to $v'_2$.  
%% Finally we got the path $v_0\rightarrow v'_1\rightarrow v''_2\rightarrow v_3$ in figure~\ref{fig:pictorial}(b) 
%% close to the green line.
%% We pick the layer where the blue line diverges first from the green line. Furthermore, get the neurons whose values are different 
%% from the green point's neuron's values. In figure ~\ref{fig:pictorial}(b), the first layer where blue line diverges from the 
%% green line is $l_2$.
%% We find all the neurons of points $v''_2$ and $v'_2$ whose values differ and mark them.     


% \begin{equation}
%     \begin{align}
%         \text{softConstrs} = \{x_5=2, x_6=0\}
%     \end{align}
% \end{equation}


%%% Local Variables:
%%% mode: latex
%%% TeX-master: "../main"
%%% End:


\section{Preliminaries}
\label{sec:model}
We define the notions and definitions in this section. 
These notions and definitions are used in the later part of the paper. 
The network is defined in the definition \ref*{def:net}. 

% l_{in}, l_{out}
\begin{df}
    \label{def:net}
    A neural network $N = (Neurons, Layers, Edges, W, B, LayerType)$ is a tuple, where
    \begin{itemize}
        \item $Neurons$ is the set of neurons in $N$,
        \item $Layers = \{l_0,...,l_k\}$ is an indexed partition of $Neurons$,
        % \item $l_i \in Layers$ represents the $i^{th}$ layer in network $N$. 
        % \item $|l_i|$ represents the number of neurons in layer $l_i$. 
        % \item 
        \item $ Edges \subseteq \Union_{i=1}^{k} l_{i-1} \times l_{i}$
        \item $W : Edges \mapsto \reals$
        \item $B : Neurons \mapsto \reals$
        \item $LayerType : Layers \mapsto \{\mathtt{Affine}, \mathtt{Relu}\}$
    \end{itemize}
\end{df}

A neural network is a collection of layers $l_0, l_1, l_2, ... l_k$, where $k$ represents the number of layers.
We call $l_0$ and $l_k$ the {\em input} and {\em output layers} respectively, and all the other layers 
as {\em hidden layers}. We assume a separate layer for activation function. Though there are different kinds of
activation, but we focus only on \relu{}, hence each layer can either be an affine layer or an \relu{} layer.
There is a one to one mapping between an affine layer $l_{i-1}$ and a \relu{} layer $l_i$. 
% If a layer $l_i$ is \relu{} layer, then the input neuron of $n_{ij}$ will only be the neuron $n_{(i-1)j}$. 
Let us say a vector $\boldsymbol{val_i} = [val_{i0}, val_{i1}, ... val_{im}]$ represents the values of each neuron 
in the layer $l_i$, where $m$ is the number of nodes in the same layer.
The value of $\boldsymbol{val_i}$ computed by the weighted sum of the previous layer's values($W_i * V_{i-1} + B_i$)
if $l_i$ is affine layer, otherwise $\boldsymbol{val_i}$ computed by the \relu{} function. 
A function $y = max(0,x)$ is a \relu{} function which takes an arguments $x$ as input and return the
same value $x$ as output if $x$ is non-negative otherwise return the value 0. 

A neural network can be visualize as a function $N$ which takes an input of $m$ dimensions and gives an 
output of $n$ dimensions. $N$ can be represented as a composition of functions $f_k o f_{l-1} ... o f_1$,
where each function $f_i$ represents either the linear combinations of previous layer's
output or an activation function. Let us say $n_{ij} \in N.Neurons$ represents 
the $i^{th}$ neuron of layer $l_j$, and $x_{ij}$ is a real variable for $n_{ij}$. 

% For any vector $\boldsymbol{v}$, $v_i$ represents it's $i^{th}$ value.  

% Let $\reals$ be the set of real numbers.
% Let $x_{\alpha}$ are unbounded set of real variables, where
% $\alpha$ is arbitrary index for variables.

\begin{df}
    \label{def:linexpr}
    $Linexpr = \{ w_0 + \sum_{i} w_i x_i | w_i \in \reals \land x_i \text{ is a real variable} \}$.
\end{df}
  
\begin{df}
    \label{def:linconstr}
    $Linconstr = \{expr \text{ op } 0 | expr \in LinExpr \land op \in \{\leq, = \}\}$
\end{df}






\begin{df}
  A matrix $W_i \in \reals^{|l_i|\times|l_{i-1}|}$ represents the weight matrix for layer $l_i$, where  
    $$
    W_i[t_1, t_2] = 
    \begin{cases}
      W(e) & e=(n_{(i-1)t_2}, n_{it_1}) \in Edges,\\
      0 & \text{otherwise.}\\
    \end{cases}
    $$
\end{df}

\begin{df}
    A matrix $B_i \in \reals^{|l_i|\times 1}$ represents the bias matrix for layer $l_i$. The entry $B_i[t,0] = B(n_{it})$, where $n_{it} \in Neurons$. 
\end{df}



\subsection{DeepPoly}
\label{sec:deeppoly}
We develop our abstract refinement approaches on top of \texttt{DeepPoly}
abstraction~\cite{deeppoly}, which is an abstraction-based method.
The abstraction uses the combination of
well understood polyhedra~\cite{polyhedra} and box~\cite{boxd} abstract domain.
The abstraction maintains
upper and lower linear expressions
as well as
upper and lower bounds for each neurons.
The variables appearing in upper and lower expressions are only from
the predecessor layer.\todo{Add some intuition}
Formally, we define the abstraction as follows. 
% Globally \texttt{DeepPoly} forms a polyhedron.
% Experimentally, it has better precision in comparison to Box~\cite{} and Zonotope~\cite{}.
% Deeppoly has the abstract transformer of various types of layers and activation functions.

\begin{df}
    For a neuron $x \in N.neurons$,
    an abstract constraint $A(x) = (lb,ub, lexpr, uexpr)$ is a tuple, where
    $lb \in \reals$ is lower bound on the value of $x$,
    $ub \in \reals$ is the upper bound on the value of  $x$,
    $lexpr \in LinExpr$ is the expression for the lower bound, and
    $uexpr \in LinExpr$ is the expression for the upper bound.
\end{df}

% The abstract constraint $A$ is generated by the tool deeppoly~\cite{} as explained in subsection~\ref*{sec:deeppoly}. 

In \texttt{DeepPoly}, we compute the abstraction $A$ as follow:
\begin{itemize}
\item For an affine neuron $x_{ij}$, we set 
  $A(x_{ij}).lexpr := A(x_{ij}).uexpr := \sum_{t=1}^{|l_{i-1}|} W[j,t]*x_{(i-1)t} + B_i[j,0]$.
  They compute the value of $A(x_{ij}).lb$ and $A(x_{ij}).ub$ by back substituting
  the variables in $A(x_{ij}).lexpr$ and $A(x_{ij}).uexpr$ respectively up to input layer.  
\item For a relu neuron $x_{ij} = max(0,x_{(i-1)j})$, we consider three cases:
            \begin{enumerate}
                \item If $A(x_{(i-1)j}).lb \geq 0$ then relu is in active phase and $A(x_{ij}).lexpr = A(x_{ij}).uexpr = x_{(i-1)j}$,
                        and $A(x_{ij}).lb = A(x_{(i-1)j}).lb$ and $A(x_{ij}).ub = A(x_{(i-1)j}).ub$
                \item If $A(x_{(i-1)j}).ub \leq 0$ then relu is in passive phase and $A(x_{ij}).lexpr = A(x_{ij}).uexpr = 0$, 
                        and $A(x_{ij}).lb = A(x_{ij}).ub = 0$
                \item  If $A(x_{(i-1)j}).lb < 0$ and $A(x_{(i-1)j}).ub > 0$ then the behavior of relu is uncertain, and authors
                        do over-approximation. $A(x_{ij}).uexpr = u(x_{(i-1)j} - l) / (u - l)$, 
                        where $u = A(x_{(i-1)j}).ub \text{ and } l = A(x_{(i-1)j}).lb$.
                        And $A(x_{ij}).lexpr = \lambda . x_{(i-1)j}$, where $\lambda \in \{0,1\}$. 
                        They are choosing the value of $\lambda$ dynamically. They compute the value of $A(x_{ij}).lb$ and $A(x_{ij}).ub$ 
                        by back substituting the variables in $A(x_{ij}).lexpr$ and $A(x_{ij}).uexpr$ respectively up to input layer. 
            \end{enumerate} 
\end{itemize}

The constraints for an affine neuron are exact because it is just an affine transformation of input neurons. 
The constraints for a relu neuron is also exact if relu is either in active or passive phase. 
The constrains for relu is over-approximated if the behavior of relu is uncertain, although the exact 
constraints for this case exist, but with the cost of efficiency. 
The equation ~\ref{eq:reluexact} shows the exact constraints
for a relu neurons when its behavior is uncertain. Let us say the relu neuron is $x_{ij} = max(0,x_{(i-1)j})$. 

\begin{align}
    \label{eq:reluexact}
    \begin{split}
        x_{ij} &\leq x_{(i-1)j} - l*(1-a) \\
        x_{ij} &\geq x_{(i-1)j} \\
        x_{ij} &\leq u*a \\
        x_{ij} &\geq 0 \\
        a &\in \{0,1\} \\ 
        \text{where }l = A(x_{(i-1)j}).lb &\text{ and }u = A(x_{(i-1)j}).ub \\
    \end{split}
\end{align}


\subsection{Solver}


%%% Local Variables:
%%% mode: latex
%%% TeX-master: "main"
%%% End:



% \clearpage

\section{Algorithm}
\label{sec:algo}
%\clearpage
A neural network is a collection of layers $l_0, l_1, l_2, ... l_k$, where $k$ represents the number of layers. 
Layer $l_0$ and $l_k$ represent the input and output layers respectively and all the other layers are hidden layers.
Each layer $l_i$ is a collection of nodes. A node $v_{i,j}$ represents the $j^{th}$ node in the layer $i$. 
Let us say a vector $\overline{V_i} = [v_{i,0}, v_{i,1}, ... v_{i,m}]$ represents the values of each node in the layer $i$, where $m$ is the number of nodes in the same layer. 
Each layer's values are computed using the weighted sum of the previous layer's values($W_i * V_{i-1} + B_i$) followed by an activation function
\relu{}. A function $y = max(0,x)$ is a \relu{} function which takes an arguments $x$ as input and return the same value $x$ as output if 
$x$ is non-negative otherwise return the value 0. 

A neural network is a function $N$ which takes an input of $m$ dimensions and gives an output of $n$ dimensions.
$N$ can be represented as a composition of functions $f_l o f_{l-1} ... o f_1$, where each function $f_i$ represents the linear combinations followed by an activation function. 

Let $\reals$ be the set of real numbers.
Let $x_{\alpha}$ are unbounded set of real variables, where
$\alpha$ is arbitrary index for variables.

\begin{df}
  $Linexpr = \{ w_0 + \sum_{i \in IndexSet} w_i x_i | w_i \in \reals \text{ and IndexSet is a finite set of indexes} \}$.
  % Let $w_i \in \reals$.
  % $w_0 + \sum w_i x_i \in Linexpr$.
\end{df}

\begin{df}
  $N = (Neurons, Edges, Layers, Input, Output )$.
  $N.neurons$?
  $n_{ij}$
  $x_{ij}$ One thing names
  $v_{ij}$ One thing values
\end{df}


\begin{df}
  For neuron $x \in N.neurons$,
  an abstract constraint $A(x) = (lb,ub, lexpr, uexpr)$ is a tuple, where
  $lb \in \reals$ is lower bound on the value of $x$,
  $ub \in \reals$ is the upper bound on the value of  $x$,
  $lexpr \in LinExpr$ is the expression for the lower bound, and
  $uexpr \in LinExpr$ is the expression for the upper bound.
  % For a neural network $N$, an abstract constraints $A : N.neurons \mapsto \reals \times \reals \times Expr \times Expr $
\end{df}

Let us say $P$ and $Q$ are the predicates on the input and output space respectively. 
The goal is to find an input $\overline{x_0}$, such that the predicates $P(\overline{x_0})$ and $Q(N(\overline{x_0}))$ holds. 
The predicate $Q$ usually is the negation of the desired property. The triple $\langle N, P, Q \rangle$ is our verification query. 

The algorithm~\ref{algo:main} represents the high level flow of our approach. Which is a counter example guided abstract refinement(CEGAR) based approach. 
At the first line in algorithm~\ref{algo:main}, we generate all the abstract constraints by \deeppoly{}. These abstract constrains
are the lower and upper constraints as well as the lower and upper bounds for each neurons in the neural network.
The \emph{isVerified} function in algorithm~\ref{algo:main} calls either algorithm~\ref{algo:verif1} or algorithm~\ref{algo:verif2}. 
Both the algorithms~\ref{algo:verif1} and \ref{algo:verif2} takes \markednewrons{} as input. The \markednewrons{} 
is a subset of the neurons of the neural network. The \markednewrons{} represents the set of the culprit neurons. 
Algorithm~\ref{algo:verif1} replace the abstract constraints of \markednewrons{} by exact constraints and check for the 
property by \milp{} solver. If \milp{} solver return \sat{} then we return satisfying assignments as a spurious counter example, 
otherwise return verified. Algorithm~\ref{algo:verif2} split each neuron of \markednewrons{} into two sub cases. Suppose the 
neuron $x\in [lb,ub]$ belongs to \markednewrons{}, then first case is when $x \in [lb,0]$ and the second case is when $x \in [0,ub]$.   
After splitting neurons into two cases \deeppoly{} run for both cases separately. In algorithm \ref{algo:verif2}, \deeppoly{}
run exponential number of times in the the size of the \markednewrons{}. We return verified in algorithm~\ref{algo:verif2} if 
verification query verified in all the \deeppoly{} runs. If \deeppoly{} fails to verify in any case then we return the spurious counter example. 


\begin{algorithm}[t]
  \textbf{Input: } A verification problem $\langle N,P,Q \rangle$ \\
  \textbf{Output: } UNSAT or SAT
  \begin{algorithmic}[1]
    \State Build abstract constrains and bounds on each neuron using deeppoly.
    \State $A$ is a set of all abstract constraints and bounds. 
    \State markedNeurons = \{\}
    \While{True}
      \State isVerified or spuriousCEX = isVerified($\langle N,P,Q \rangle$ , A, markedNeurons)
      \If{isVerified is True}
        \State \textbf{return} UNSAT
      \Else
        \If{spuriousCEX is a real cex of $\langle N,P,Q \rangle$}
          \State \textbf{return} spuriousCEX
        \Else
          \State marked, cex = getMarkedNeurons($\langle N,P,Q \rangle$ , A, markedNeurons)
          \If{cex not None}
            \State \textbf{return} cex
          \EndIf
          \State markedNeurons = markedNeurons $\union$ marked
        \EndIf
      \EndIf
    \EndWhile
  \end{algorithmic}
  \caption{A CEGAR based approach of neural network verification}
  \label{algo:main}
\end{algorithm}

\begin{algorithm}[t]
  \textbf{Name: } isVerified1 \\
  \textbf{Input: } $\langle N,P,Q \rangle$ with abstract constraints $A$ and markedNeurons $\subseteq ~ N.neurons$ \\
  \textbf{Output: } verified or spurious counter example. 
  \begin{algorithmic}[1]
    \State constr = \{\}
    \For{$i=1$ to $k$}
      \For{$j=0$ to $l_i.size$}
        \If{$n_{i,j} \in $ markedNeurons}
          \State constr.add(exactConstr($n_{i,j}$)) 
        \Else
          \State constr.add($A(n_{i,j},lexpr) \leq x_{i,j} \leq A(n_{i,j},uexpr)$)
        \EndIf
      \EndFor
    \EndFor
    \State $constr = constr ~ \union ~ P ~ \union ~ \neg Q$
    \State isSat = checkSat(constr)
    \If{isSat is True}
      \State \textbf{return} spurious counter example
    \Else
      \State \textbf{return} verified
    \EndIf
  \end{algorithmic}
  \caption{An approach to verify $\langle N,P,Q \rangle$ with abstraction A}
  \label{algo:verif1}
\end{algorithm}

\begin{algorithm}[t]
  \textbf{Name: } isVerified2 \\
  \textbf{Input: } $\langle N,P,Q \rangle$ with abstract constraints $A$ and markedNeurons \\
  \textbf{Output: } verified or spurious counter example. 
  \begin{algorithmic}[1]
    \For{all combination in $2^{markedNeurons}$}
      \State $A'$ = run deeppoly
      \If{not verified by deeppoly}
        \State constr = \{\}
        \For{$i=1$ to $k$}
          \For{$j=0$ to $l_i.size$}
            \State constr.add($A'(n_{i,j},lexpr) \leq x_{i,j} \leq A'(n_{i,j},uexpr)$)
          \EndFor
        \EndFor
        \State $constr = constr ~ \union ~ P ~ \union ~ \neg Q$ 
        \State isSat = checkSat(constr)
        \If{isSat}
          \State \textbf{return} spurious counter example
        \EndIf
      \EndIf
    \EndFor
    \State \textbf{return} verified
  \end{algorithmic}
  \caption{An approach to verify $\langle N,P,Q \rangle$ with abstraction A}
  \label{algo:verif2}
\end{algorithm}



\begin{algorithm}[t]
  \textbf{Name: } getMarkedNeurons1 \\
  \textbf{Input: } $\langle N,P,Q \rangle$ with abstract constraints $A$ and satisfying assignments $\overrightarrow{x}$ and $\overrightarrow{y}$ on input and output layers respectively\\
  \textbf{Output: } marked neurons or real counter example. 
  \begin{algorithmic}[1]
   \State Run neural network on $\overrightarrow{x}$.
   \State \textbf{return} $\overrightarrow{x}$ as counter example if it's output violate the $Q$. 
   \For{$j=0$ to $l_k.size$}
    \State $var_{k,j} = \overrightarrow{y}_j$
   \EndFor
   \For{$i=k$ to $1$}
      \If{$l_i$ is affine layer}
        \State layerConstraints = \{\}
        \For{$j=0$ to $l_i.size$}
          \State $constr_{i,j}$ = $(A(x_{i,j},lexpr)$ == $var_{i,j}$) \Comment{lexpr=rexpr for affine layer}
          \State layerConstrains.add($constr_{i,j}$)
        \EndFor
        % \For{$j=0$ to $l_{i-1}.size$}\Comment{m is size of layer $l_{i-1}$}
        %   \State layerConstrains.add$(A(x_{i,j}, lb))$
        %   \State layerConstrains.add$(A(x_{i,j}, ub))$
        % \EndFor
        \State isSat = checkSat(layerConstrains) \Comment{And of all constraints}
        \If{isSat}
          \State assign sat values to previous layers neurons
        \Else
          \State markedNeurons = \{$x_{i,j}$ | $0 \leq j\leq l_i.size \land constr_{i,j}$ $\in$ unsatCore\}
          \State \textbf{return } markedNeurons
        \EndIf
      \Else \Comment{Relu layer}
        \For{$j=0$ to $l_i.size$}
          \If{$var_{i,j} > 0$}
            \State $var_{i-1,j} = var_{i,j}$
          \Else
            \State $A(x_{i-1,j},lb) \leq var_{i-1,j} \leq$ 0 \Comment{lb,ub are bounds}
          \EndIf
        \EndFor
      \EndIf
   \EndFor
    \State ce = \{$var_{0,i}$ | $i=0$ to $l_{0}.size$ \} \Comment{If pullbacked upto input layer}
    \State \textbf{return} ce
  \end{algorithmic}
  \caption{A pullback approach to get mark neurons or counter example}
  \label{algo:refine1}
\end{algorithm}


\begin{algorithm}[t]
  \textbf{Name: } getMarkedNeurons2 \\
  \textbf{Input: } $\langle N,P,Q \rangle$ with abstract constraints $A$ and satisfying assignments $\overrightarrow{x}$ and $\overrightarrow{y}$ on input and output layers respectively\\
  \textbf{Output: } marked neurons or real counter example. 
  \begin{algorithmic}[1]
    \State Run neural network on $\overrightarrow{x}$.
    \State \textbf{return} $\overrightarrow{x}$ as counter example if it's output violate the $Q$. 
    \State Let us say $\overrightarrow{val_{i}}$ is the value evaluated on $\overrightarrow{x}$ on layer $l_i$. 
    \For{$i=1$ to $k$} \Comment{inputLayer excluded}
      \If{$l_i$ is affine layer}
        \State $\overrightarrow{val_i} = W_i * \overrightarrow{val_{i-1}} + B_i$ \Comment{simple matrix muliplication}
      \Else
        \State constraints = \{\}
        \For{$t=i$ to $k$}\Comment{Each layer after $l_i$}
          \For{$j=0$ to $l_t.size$}
            \State constrains.add($A(var_{t,j},lexpr) \leq var_{t,j}\leq A(var_{t,j},uexpr)$)
          \EndFor
        \EndFor
        \For{$j=0$ to $l_{i-1}.size$}\Comment{Previous layer}
          \State constrains.add($var_{i-1,j} == val_{i-1,j}$)
        \EndFor
        \For{$j=0$ to $l_k.size$}\Comment{Output layer}
          \State constrains.add($var_{k,j} == val_{k,j}$)
        \EndFor
        softConstraints = \{\}
        \For{$j=0$ to $l_i.size$}\Comment{Current layer}
          \State softConstraints.add($var_{i,j} == val_{i,j}$)
        \EndFor
        \State maximize the softConstraints with satisfying the constrains. 
        \If{all the softConstraints satisfied}
          \State \textbf{continue}
        \Else
          \State markedNeurons = all neurons of layer $l_i$ which does not satisfy the softConstraints. 
          \State \textbf{return} markedNeurons
        \EndIf 
      \EndIf
    \EndFor
  \end{algorithmic}
  \caption{An optimization based approach to get mark neurons or counter example}
  \label{algo:refine2}
\end{algorithm}






%--------------------- DO NOT ERASE BELOW THIS LINE --------------------------

%%% Local Variables:
%%% mode: latex
%%% TeX-master: "main"
%%% End:


%\section{Variations}
%\label{sec:variations}
% \begin{figure}[t]
    \centering
    \begin{tikzpicture}[level/.style={sibling distance=60mm/#1}]
        \node(g){$\globally$} child {node (u) {$\until$} child {node (s) {$\varphi(2)$} child[dashed, gray]
            {node[draw, circle, minimum size=0.3cm, gray] (l) {} child
            {node[draw, circle, minimum size=0.3cm, gray] (ll) {}} child
            {node[draw, circle, minimum size=0.3cm, gray] (lr) {}}}
            child[dashed, gray] {node[draw, circle, minimum size=0.3cm] (r) {}
            child {node[draw, circle, minimum size=0.3cm, gray] (rl)
            {}} child {node[draw, circle, minimum size=0.3cm, gray]
            (rr) {}}}} child {node
            (x) {$x$}}};
    \end{tikzpicture}
    \caption{Example subtree construction for Hybrid Pattern Matching}
\label{fig:subtree}
\end{figure}

We have presented a scheme of ranking formulae and an algorithm to find the best
formula ranked on a trace.
%
However, our scheme has many possibilities of variations. For example, the
choice of various constants and relative importance of LTL operators is one
such. 
%
In this section, we will discuss a couple of variations.
% \sankalp{Possibly an example involving several equally scored formulae? Thread
% locking? th1\_wait U $\neg$ lock, as well as th2\_wait U $\neg$ lock?}

\emph{Hybrid Pattern Matching: }
The algorithm as defined in  sections \ref{sec:cso} and \ref{sec:opm} suffers
from a certain common drawback, though at different ends of the spectrum.
%
Otherwise, we can always discover more and more complex formulae.
%
The issue is further enhanced by the constraint sizes growing far too  quickly
with increase in the depth of the pattern. 

In the case of  \emph{pattern matching}, the algorithm does not check a large
enough variety of properties (it is guided by the template pattern) and, for the
same reason,  is also quite \emph{insufficiently} expressive in comparison with
constrained system optimization. 
%
To remedy this, we allow a middle ground, where, instead of attempting to learn
formulae from scratch or from explicit patterns, we allow learning
\emph{subformulae} within some known patterns.
%
A subformulae argument $\varphi(n)$ with $n$ being a prescribed maximum depth for
the subtree can be provided as part of the pattern, parsed into the tree as an
abstracted empty formula with constraints constructed for the specified nodes
explicitly, and for the subformulae recursively in the manner as described for
seciton \ref{sec:cso}. In Figure \ref{fig:subtree}, we show an example of the
hybrid pattern $\globally (\varphi(2) \: \until \: x)$, where $\varphi(2)$ is an
unknown formula of depth less than $2$ and $x$ is an unknown letter.

\begin{algorithm}
    \begin{algorithmic}[1]
        \Procedure{HybridPattern}{$P, N, pattern$} \Comment{Returns optimal
            formula fitting a pattern}
        
            \State \textbf{parse} pattern 
            \State construct $\varphi^{\var}_\prop$ for propositional patterns in tree
            \State construct $\varphi^{ST}_\textnormal{n}$ for every subformula pattern $\varphi(n)$ in the tree
            \State constraint $\Phi \leftarrow \varphi^{\var}_\prop \land \bigwedge \varphi^{ST}_\textnormal{n}$
            \State optimize
            $\min( \{y_{0,0}^\tau\: |\: \tau \in P\} )$ with $\Phi$ as constraint
            
            \If{optimization succeeds with model m} \Comment SAT \State
                construct formula tree from m \State \textbf{return}
                optimized formula tree \Else \State \textbf{return} UNSAT \EndIf
                \EndProcedure
    \end{algorithmic}
    \caption{Computing the optimal formula given a partial pattern}
    \label{algo:hybrid}
\end{algorithm}

\emph{Prioritize variables: }
%
In case there is a large set of events and we want bias the focus of our search
towards certain letters in the traces that do not occur very often, we may
adjust the value $V(p,w)$ assigned to each propositional variable $p$.
%
In our default scheme, we assign $V(p,w) = 1$ if $w(1)$ contains $p$.
%
A user may assign a value greater than $1$ to variables $p$ that are desirable
and assign less than $1$ for the variables $p$ that are not.
% %
% As a variation of Optimized Pattern Matching, we let the user to fix a subset
% of  variables to appear in the pattern and compute the best mapping of the
% other variables per specification.
%
This allows for mining specifications pertaining to the prioritized variables in
cases where several competing well ranked specifications are present.
%
Let $\pi$ be the map from the propositional variables $\prop$ to their priority.
%
We replace equation~\eqref{eq:prop_score} by the following formula where we
return score $\pi(p)$ instead of $1$. 
% %
% To achieve this, in addition to~\eqref{eq:map}, the following
% constraint~\eqref{eq:fixed} is added, which binds the common variable to
% itself.
% %
% \begin{align} \bigwedge _{p \in Var \intersection \prop} m_{p, p}
%     \label{eq:fixed} \end{align}

\begin{align}
    \bigwedge _{1 \leq i \leq N} \bigwedge _{p \in \prop} x_{i, p} \rightarrow \left[ 
        \bigwedge _{1 \leq t \leq |\tau|} y^\tau _{i, t} = \begin{cases}
            \pi(p) & \textnormal{if } p \in \tau (t) \\
            0 & \textnormal{if } p \not \in \tau (t)
        \end{cases}
    \right] \label{eq:prop_priority_score} 
\end{align}

Recall that we considered the trace in figure~\ref{fig:exampletrace}, where we
wanted to find a property $\globally \:\texttt{auth\_failed}\: \Leftrightarrow\:
\varphi(1)$. We may further abstract the property $\globally \: \varphi(0) \:
\Leftrightarrow\: \varphi(1)$ by setting the priority map $\pi =
\{\texttt{auth\_failed} \mapsto 10, otherwise \mapsto 1\}$. We obtain the same
result from \ourtool.

% Of course, the best mapping for $x$ is $\texttt{connected}$ and the tool
% returns the formula $\globally (\texttt{connected} \until
% \texttt{disconnected})$. 

The above variations suggest that our method is viable to be adapted  to an
application at hand, where we want to bias our ranking to give preference to a
desired class of formulae.


% --- no delete below this line --

%%% Local Variables:
%%% mode: latex
%%% TeX-master: "main"
%%% End:


%\clearpage
\section{Experiments}
\label{sec:experiments}
We have implemented our approach in a prototype and compared it three types of related approaches (i) \deeppoly{}~\cite{singh2019abstract} and its refinements \kpoly{}~\cite{singh2019beyond}, \deepsrgr{}~\cite{yang2021improving}, (ii) other cegar based approaches and 
%The approaches  \texttt{kPoly} and \texttt{deepSRGR} do the refinement on \deeppoly{}.
(iii) state-of-the-art tools \alphabeta~\cite{zhang2018efficient,wang2021beta,xu2020fast,zhang2022branch,tjeng2017evaluating} and 
\ovaltool~\cite{bunel2018unified,bunel2020branch,bunel2020lagrangian,de2021scaling,de2021scaling,de2021scaling2,de2021improved}, both of which use a set/portfolio of different algorithms and optimizations. The first tool~\alphabeta{} achieved the 1st and \ovaltool{} 3rd rank in the 2nd International Verification of Neural Networks Competition (VNN-COMP'21).
%We use the MNIST \cite{deng2012mnist} for our evaluation.    
\subsection{Implementation}
We have implemented our techniques in a tool, which we call \drefine{}, in \textsc{C++} programming language. Our approach relies on \deeppoly{}, so we also have implemented \deeppoly{} in \textsc{C++}. We are using a \textsc{C++} interface of the tool Gurobi~\cite{gurobioptimizer} to check the satisfiability as well as solve  \maxsat{} queries. %We have implemented our technique as a tool and call it \drefine{}.  

\subsection{Benchmarks}
We use the MNIST~\cite{deng2012mnist} dataset to check the effectiveness of our tool and comparisons. We are using 11 different fully connected feedforward neural networks with $\relu${} activation as shown in table~\ref{tb:nndetail}.
These benchmarks are taken from the \deeppoly{}'s paper~\cite{singh2019abstract}.  The input and output dimensions of each network are $784$ and $10$ respectively. 
The authors of \deeppoly{} used projected gradient descent (PGD)~\cite{dong2018boosting}
and DiffAI~\cite{mirman2018differentiable} for adversarial training. Table \ref{tb:nndetail} contains the defended network i.e.
trained with adversarial training as well as the undefended network. The last column of the table \ref{tb:nndetail} how the defended networks were trained.  

The predicate $P$ on the input layer is created using the input image $\boldsymbol{im}$ and user-defined parameter $\epsilon$.  We first normalize each pixel of $\boldsymbol{im}$ between $0$ and $1$, then create  $P = \Land_{i=1}^{|l_0|} im(i)-\epsilon \leq x_{0i}\leq im(i)+\epsilon$, such that the lower and upper bound of each pixel should not exceed $0$ and $1$ respectively. The predicate $Q$ on the output layer is created using the network's output.     Suppose the predicted label of $\boldsymbol{im}$ on network $N$ is $y$, then $Q = \Land_{i=1}^{|l_k|} x_{ki} < y$, where $i \neq y$.  One query instance $\langle N,P,Q \rangle$ is created for one network, one image and one epsilon value.  In our evaluation, we took $11$ different networks, 8 different epsilons, and 100 different images. The 
total number of instances is $8800$. However, there are 128 instances for which the network's predicted label differs from the image's actual label. we avoided such instances, so, there is a total of $8672$ benchmark instances
under consideration.    Whenever our tool found a counter-example on $\langle N,P,Q \rangle$, it denormalizes it into an image by rounding the float values 
and checks for counter-example by executing $N$ on the denormalized image.
If a counter-example is found then the tool reports it, otherwise, the tool reports unknown.



\begin{table}[t]
    \centering
    \begin{tabular}{c|c|c|c}
        \hline
        \textbf{Neural Network} & \textbf{\#hidden layers} & \textbf{\#activation units} & \textbf{Defensive training} \\
        \hline
        $3\times 50$ & 2 & 110 & None \\
        $3\times 100$ & 2 & 210 & None  \\
        $5\times 100$ & 4 & 410 & None  \\
        $6\times 100$ & 5 & 510 & DiffAI \\
        $9\times 100$ & 8 & 810 & None  \\
        $6\times 200$ & 5 & 1010 & None  \\
        $9\times 200$ & 8 & 1610 & None  \\
        $6\times 500$ & 6 & 3000 & None  \\
        $6\times 500$ & 6 & 3000 & PGD, $\epsilon = 0.1$ \\
        $6\times 500$ & 6 & 3000 & PGD, $\epsilon = 0.3$ \\
        $4\times 1024$ & 3 & 3072 & None  \\
        \hline
    \end{tabular}
    \caption{Neural networks details}
    \label{tb:nndetail}
\end{table}

\subsection{Results}
We conducted the experiments on a machine with \textsc{64GB RAM, 2.20 GHz Intel(R) Xeon(R) CPU E5-2660 v2}
processor with CentOS Linux 7 operating system. 
To make a fair comparison between the tools, we provide only a single \textsc{CPU} for each instance for each tool. 
We make three types of comparisons as shown in (i) Figure~\ref{res:milp_with_milp}, which is a cactus plot of (log of) time taken vs the number of benchmarks solved (ii) Table~\ref{tb:matrix}, which makes a pairwise comparison of the number of instances that a tool could solve which another couldn't  (more precisely, the $(i,j)$-entry of the table is the number of instances which could be verified by tool $i$ but not by tool $j$) and (iii) Figure~\ref{res:ep:milp_with_milp} which compares wrt epsilon, the robustness parameter.
%represents the verified cases while the column represents the not verified cases, i.e.\kpoly{}'th row and \deeppoly{}'th column represent the  156 number of benchmarks instances which are solved by \kpoly{} and not solved by \deeppoly{}. 

\paragraph{Comparison with the most related techniques:}
In this subsection, we consider the techniques \deeppoly{}, \kpoly{}, and \deepsrgr{} to compare with our technique. 
We consider \deeppoly{} because it is at the base of our technique, and the techniques \kpoly{} and \deepsrgr{} refine \deeppoly{} just as we do. These tools only report \verified{} instances, while our tool can report  \verified{} and the counter-example. We note that we compare these techniques with only \verified{}  instances of our technique in the line of \textsc{drefine\_verified} in cactus plot ~\ref{res:milp_with_milp}. 

It can be seen that our technique outperforms the others in terms of the verified number of instances. One can also see that when they do verify, \deeppoly{} and \kpoly{} are often more efficient, which is not surprising, while our tool is more efficient than \deepsrgr{}. From Table~\ref{tb:matrix}, we also see  that our tool solves all the benchmark instances which are solved by these three techniques (and significantly more: $\sim 3700$ for the first two and $\sim 1000$ for the third), except $14$ instances where \kpoly{} succeeds and our tool times out.

% The approaches \deeppoly{}, \texttt{refinepoly}, and \texttt{deepSRGR} are incomplete and do not report the counter-example. 
% Our approach is a complete approach. We compare our approach's \texttt{verified} only instances with incomplete approaches. 
% Figure~\ref{res:milp_with_milp} shows the comparison with the help of the cactus plot. Our approach performs well 
% in terms of the verified number of instances as well as in terms of time in comparison to the above incomplete approaches. 
% We are also comparing with state of the arts to check the effectiveness of our approach globally.
% Table~\ref{tb:soacomparison} shows that our tool is verifying $180$ and $190$ benchmarks which $\alpha -\beta$-CROWN and 
% \texttt{oval} respectively are not able to verify. In addition to this, our tool verifies $172$ unique instances 
% that neither $\alpha -\beta$-CROWN nor \texttt{oval} is able to verify.  
% Despite it, $\alpha - \beta-$Crown and \texttt{oval} are verifying more benchmarks in comparison to our tool. 
% Although the total time taken by both state of the arts is higher than our tool as shown in figure~\ref{res:milp_with_milp}. 


% \begin{table}
%     \centering
%     \begin{tabular}{|c|c|c|c|}
%         \hline
%         -  & Not verified by oval & Not verified by $\alpha - \beta$-CROWN & Not verified by both \\
%         \hline
%         Verified by our tool & 190 & 180 & 172 \\
%         \hline
%     \end{tabular}
%     \caption{Comparison with $\alpha - \beta$-CROWN and Oval}
%     \label{tb:soacomparison}
% \end{table}



\begin{figure}[t]
  % \centering
  
\begin{tikzpicture}
    \begin{axis}[
        xlabel={Number of benchmarks},
        ylabel={log(aggregated time)},
        width=7cm,
        height=6cm,
        xmin=0, xmax=6000,
        ymin=0, ymax=25,
        xtick={0, 2000, 4000 ,6000},
        ytick={0,5,10,15,20,25},
        legend pos=south east,
        legend entries=  {\tiny \deeppoly{}, \tiny \kpoly{}, \tiny \deepsrgr{}, \tiny \textsc{drefine\_verified}, \tiny \drefine{}},%, \tiny \drefine{}, \tiny \alphabeta{}, \tiny \ovaltool{}, \tiny \cegarnn{}, \tiny \marabou{}},
        ymajorgrids=true,
        xmajorgrids=true,
        grid style=dashed,
    ]
    
    \addplot[
        color=cyan
    ] 
    table {fig/deeppoly_data.txt};

    \addplot[
        color=green
    ] 
    table {fig/refinepoly_data.txt};

    \addplot[
        color=orange
    ] 
    table {fig/deepsrgr_data.txt};


    \addplot[
        color=violet
    ]
    table {fig/drefine_verified_data.txt};

    \addplot[
        color=blue
    ] 
    table {fig/drefine_data.txt};
    
    % \addplot[
    %     color=purple
    % ]
    % table {fig/alpha_beta_data.txt};

    % \addplot[
    %     color=brown
    % ]
    % table {fig/oval_data.txt};

    % \addplot[
    %     color=gray
    % ]
    % table {fig/four_class_data.txt};

    % \addplot[
    %     color=black
    % ]
    % table {fig/marabou_data.txt};

    \end{axis}

\end{tikzpicture}
  \caption{Cactus plot with related techniques}
  \label{res:milp_with_milp}
\end{figure}

\paragraph{Comparison with cegar based techniques: }
\cegarnn{}~\cite{elboher2020abstraction} is a tool that also uses  counter example guided refinement. But the abstraction used is quite different in comparison to the  \deeppoly{}. This tool reduces the size of the network by merging the similar neurons, such that they maintain the overapproximation and split back in the refinement process. We can see in Figure~\ref{res:milp_with_milp} that \cegarnn{} verified only  $18.88\%$ while our tool verified $61.42\%$ of the total number of benchmark instances. Although, in total \cegarnn{} solves significantly fewer benchmarks, it is pertinent to note that this technique solves many unique benchmark instances as can be inferred from Table~\ref{tb:matrix}. Again this shows the orthogonal nature of this technique compared to the others.% the tool \texttt{cegar\_nn} is deployed completely different technique in comparison to others. 


\paragraph{Comparison with state-of-the-art solvers: }
The tools \alphabeta{} and \ovaltool{} use several algorithms that are highly optimized and use several techniques. The authors of \alphabeta{} implement the techniques~\cite{zhang2018efficient,wang2021beta,xu2020fast,zhang2022branch,tjeng2017evaluating}, 
and the authors of \ovaltool{} implement~\cite{bunel2018unified,bunel2020branch,bunel2020lagrangian,de2021scaling,de2021scaling,de2021scaling2,de2021improved}
Figure~\ref{res:milp_with_milp} shows that both the tools indeed solve about 1000 (out of 8000) more than we do. However, we found around $180$ benchmarks instances where \alphabeta{} fails and our tool works. Also, around $190$ benchmarks on which \ovaltool{} fails and our tool works; see Table~\ref{tb:matrix} for more details. In total, we are solving $172$ unique benchmarks where both tools fail to solve. Thus we believe that these tools are truly orthogonal in their strengths and could potentially be combined. We also note that in terms of total time, both tools take more time than our tool. 

\begin{figure}[t]
%    \centering
\begin{tikzpicture}
    \begin{axis}[
        xlabel={epsilons},
        ylabel={percentage of verified/falsified tasks},
        width=11cm,
        height=6.5cm,
        xmin=0.005, xmax=0.05,
        ymin=0, ymax=120,
        xtick={0.005,0.01,0.015,0.02,0.025,0.03,0.04,0.05},
        ytick={0,20,40,60,80,100},
        legend pos=north east,
        legend entries={\tiny \drefine{}, \tiny \textsc{deeppoly}, \tiny \textsc{drefine\_verified}, \tiny \textsc{deep\_srgr}, \tiny \textsc{kpoly}, \tiny \textsc{$\alpha\beta$-crown}, \tiny \textsc{oval}, \tiny \cegarnn{}},
        ymajorgrids=true,
        grid style=dashed,
    ]
    \addplot+[
        color=blue,
        mark=*,
    ]
    coordinates {
        (0.005,99.72)(0.01,89.39)(0.015,73.98)(0.02,65.03)(0.025,51.84)(0.03,45.01)(0.04,32.38)(0.05,30.9)
    };

    \addplot[
        color=gray,
        mark=*,
    ]
    coordinates {
        (0.005,99.07)(0.01,86.9)(0.015,68.91)(0.02,58.3)(0.025,43.35)(0.03,33.58)(0.04,20.38)(0.05,16.89)
    };

    \addplot[
        color=violet,
        mark=*,
    ]
    coordinates {
        (0.005,99.07)(0.01,87.73)(0.015,71.58)(0.02,61.62)(0.025,50.92)(0.03,40.12)(0.04,25.55)(0.05,20.84)
    };

    \addplot[
        color=orange,
        mark=*,
    ]
    coordinates {
        (0.005,99.8)(0.01,87.26)(0.015,69.83)(0.02,58.57)(0.025,44.37)(0.03,34.13)(0.04,20.75)(0.05,17.06)
    };

    \addplot[
        color=green,
        mark=*,
    ]
    coordinates {
        (0.005,99.26)(0.01,87.08)(0.015,70.94)(0.02,60.42)(0.025,46.67)(0.03,37.63)(0.04,21.78)(0.05,17.99)
    };

    \addplot[
        color=purple,
        mark=*,
    ]
    coordinates {
        (0.005,99.9)(0.01,95.29)(0.015,82.02)(0.02,71.05)(0.025,64.60)(0.03,60.09)(0.04,51.98)(0.05,50.96)
    };

    \addplot[
        color=brown,
        mark=*,
    ]
    coordinates {
        (0.005,99.9)(0.01,95.11)(0.015,82.1)(0.02,71.21)(0.025,64.76)(0.03,59.96)(0.04,50.92)(0.05,48.8)
    };

    \addplot[
        color=black,
        mark=*,
    ]
    coordinates {
        (0.005,17.98)(0.01,18.45)(0.015,19.71)(0.02,19.92)(0.025,21.22)(0.03,19.92)(0.04,17.95)(0.05,16.23)
    };


    \end{axis}

\end{tikzpicture}
%%% Local Variables:
%%% mode: latex
%%% TeX-master: t
%%% End:

    \caption{Size of input perturbation (epsilon) vs. percentage of solved instances}
    \label{res:ep:milp_with_milp}
\end{figure}


\begin{table}[t]
  \footnotesize
    \centering
    \begin{tabular}{|c|c|c|c|c|c|c|c||c|}
        \hline
        \diagbox{\tiny Verified}{\tiny Not verified} & \tiny \textbf{\cegarnn} & \tiny \textbf{\deeppoly} & \tiny \textbf{\kpoly} & \tiny \textbf{\deepsrgr} & \tiny \textbf{\alphabeta} & \tiny \textbf{\ovaltool} & \tiny \textbf{\drefine} & \tiny \textbf{\textsc{total}} \\
        \hline
        \tiny \textbf{\cegarnn} & \textbf{0} & 713 & 609 & 687 & 217 & 238 & 417 &  1638 \\ 
        \hline
        \tiny \textbf{\deeppoly} & 3708 & \textbf{0} & 0 & 0 & 63 & 66 & 0  & 4633 \\ 
        \hline
        \tiny \textbf{\kpoly} & 3760 & 156 & \textbf{0} & 114 & 63 & 66 & 14  & 4789  \\ 
        \hline
        \tiny \textbf{\deepsrgr} & 951 & 54 & 2 & \textbf{0} & 63 & 66 & 0  & 4687 \\ 
        \hline
        \tiny \textbf{\alphabeta} & 4821 & 1672 & 1516 & 1618 & \textbf{0} & 84 & 1095  & 6242 \\
        \hline
        \tiny \textbf{\ovaltool} & 4789 & 1622 & 1466 & 1568 & 31 & \textbf{0} & 1052 & 6189  \\
        \hline
        \tiny \textbf{\drefine} & 4106 & 694 & 552 & 640 & 180 & 190 & \textbf{0} & 5327 \\
        \hline
    \end{tabular}
    \caption{Comparison of tools, e.g. the entry on row \kpoly{} and column \deeppoly{} represents 156 benchmark instances on which \kpoly{} verified but \deeppoly{} fails}
    \label{tb:matrix}
\end{table}

\paragraph{Epsilon vs. performance: }
For sanity check, we analyzed the effect of perturbation size and the performance of the tools.
In Figure~\ref{res:ep:milp_with_milp}, we present the comparison of fractional success rate of tools as the size of epsilon values grow from $0.005$ to $0.05$. 
At $\epsilon=0.005$, the performance of all the tools is almost the same. As the value of epsilon increases, the success rate of the tools drop consistently.
Here also, we perform better than the approaches \deeppoly{}, 
\kpoly{}, \deepsrgr{} and \cegarnn{}, while \alphabeta{} and \ovaltool{} perform better.


\paragraph{Number of marked neurons: }
On an average, \drefine{} marks $9.8\%$ of neurons, whenever it needs to refine. 
Our experiments show that the number of marked neurons is quite small, so our refinement is indeed fast and strengthens the argument that {\em "finding causes of spuriousness"} does improve the overall efficiency. 

% Figures \ref{res:milp_with_milp} and \ref{res:ep:milp_with_milp} shows that we are performing better in compare to
% the most related techniques \deeppoly{}, \texttt{refinepoly} and \texttt{deepSRGR}. 


% \todo{recall what is epsilon... x axis label can perhaps be written out more}



%--------------------- DO NOT ERASE BELOW THIS LINE --------------------------

%%% Local Variables:
%%% mode: latex
%%% TeX-master: "../main"
%%% End:



\section{Related work}
\label{sec:related}
The papers also do the refinement on \deeppoly{} are \texttt{deepSRGR}~\cite{yang2021improving} and
\texttt{refinepoly}~\cite{singh2019beyond}, 
although these papers do not do the cegar-based refinement. The approach \texttt{refinepoly} considers 
a group of neurons at once to generate the constraints and compute the bounds of neurons. 
The approach \texttt{deepSRGR} removes the 
spurious region by taking each neuron's constraints with the negation of the property and using the 
\texttt{Gurobi}~\cite{gurobioptimizer} optimizer to tighten the bounds of each neuron. 
As per our knowledge elbohar et al~\cite{elboher2020abstraction} and \texttt{NARv}~\cite{liu2022abstraction} 
do the cegar-based refinement, but their abstraction techniques are orthogonal to \deeppoly{}. 
They reduce the network size by merging similar 
neurons with over-approximation, while \deeppoly{} maintains the linear constraints for each neuron without changing the 
structure of the network. Paper~\cite{lin2020art} do the refinement by bisecting the input space on the guided dimension. 
Our approach exploits the incomplete technique \deeppoly{},
which scales well, but if it fails to verify then we do the cegar-based refinement.

%  In general neural network verification techniques can be classify broadly in two categories, 
%  complete and incomplete techniques. The techniques \cite{lomuscio2017approach}, \cite{fischetti2018deep},
%  \cite{dutta2018output}, \cite{cheng2017maximum}, \cite{katz2017reluplex}, \cite{katz2019marabou}, 
%  \cite{ehlers2017formal}, \cite{huang2017safety}, \cite{wang2021beta}, \cite{xu2020fast}, \cite{zhang2022general}
%  are the complete techniques. Techniques which either return \texttt{verified} or a counter example 
%  are known as complete techniques. 
%  The complete techniques explore the exact state space, hence, they suffers with scalability issues on the large 
%  size of networks. On the other hand, the techniques \cite{dvijotham2018dual}, \cite{gehr2018ai2}, \cite{singh2018fast},
%  \cite{singh2018boosting}, \cite{weng2018towards}, \cite{wong2018provable}, \cite{zhang2018efficient}, \cite{zhang2018efficient}
%  are the sound and incomplete techniques. The incomplete techniques usually over approximate the state space, hence, 
%  these techniques scale well but do not remain complete. Our approach exploit the incomplete technique \deeppoly{},
%  which scale well, but if it fails to find the counter example then we do the cegar based refinement.   



\section{Conclusion and Future work}
\label{sec:conclusion}
%We have presented a novel cegar-based approach, which results in a scalable and efficient approach.  
Our approach comprises two parts. One part finds the causes of spuriousness, 
while the other parts refine  the information found in the first part. 
Experimental evaluation shows that we outperform related refinement techniques, in terms of efficiency and effectivity. 
We are also verify several benchmarks that are beyond state of the art solvers, highly optimized solvers.
As futurework, we plan to extend our technique/tool to make it independent of \deeppoly{} and applicable with other 
abstraction based techniques and tools. %We believe this could lead to more benchmarks being solved overall.
 %Such that we can improve the results on top of the state of the arts.  

% Our experiments further emphasize that the DREFINE technique we use is orthogonal to most other techniques. 
% Specifically, we utilize counterexamples to identify the intrinsic cause of spuriousness, 
% such as marked or branching neurons, whereas existing methods do not analyze counterexamples to identify the neurons to branch. 
% Consequently, it is challenging to identify the exact patterns or features present in the unique benchmarks that 
% \drefine{} solves and that current state-of-the-art methods cannot solve.

% The neural network verification currently works only on small networks. However, similar to software verification, 
% where each paper contributed a small amount towards practical software verification technology, 
% we expect to make gradual progress towards practical neural network verification technology. 
% The process of discovering new methods to solve the problem and identifying which ones will lead 
% to a push-button technology is hard to predict. 
% However, as with software verification, we may develop techniques that can solve sub-problems or neighboring problems, 
% which can make significant contributions to the area.
% Experience in SAT-solving suggests that one stack of techniques is unlikely to solve all instances. 
% Therefore, we need portfolio solvers, and we contend that our technique contributes to building such a solver.



\bibliographystyle{unsrt}
\bibliography{biblio}

%\appendix

%% \onecolumn

\begin{center}
    \Large{\bf{Appendix}}    
\end{center}

\section{Experiments}
\subsection{Comparison with PGD's adversarially trained network}
\label{ap:exp:1}

The networks considered for evaluation in this study are the ones corresponding to the 9th, and 10th 
rows of Figure~\ref{tb:nndetail}. These networks have been trained using adversarial techniques, 
where adversarial examples were generated using PGD attack, and the network was subsequently 
trained on these adversarial examples to enhance its robustness. Figure~\ref{tb:matrix2} presents a 
pairwise comparison of verifiers on these adversarially trained networks, 
encompassing a total of 1456 benchmark instances. 
Our approach demonstrates significant superiority over \deeppoly{}, \kpoly{}, and \deepsrgr{} in terms of 
performance on these benchmarks. While \alphabeta{} and \ovaltool{} outperform our approach by approximately 
90 benchmarks, we are still able to solve 75 unique benchmarks that remain unsolved by both of these tools. 
Considering the total number of benchmarks is 1456, this indicates a notable number of benchmarks that our 
approach successfully addresses.

\begin{figure}[t]
  \scriptsize
    \centering
    \begin{tabular}{|c|c|c|c|c|c|g||c|}
        \hline
        \diagbox{\tiny Verified}{\tiny \mbox{}\hspace{-2mm}Unverified} & \tiny \textbf{\deeppoly} & \tiny \textbf{\kpoly} & \tiny \textbf{\deepsrgr} & \tiny \textbf{\alphabeta} & \tiny \textbf{\ovaltool} & \tiny \textbf{\drefine} & \tiny \textbf{\textsc{total}} \\
        \hline
        \tiny \textbf{\deeppoly} & \textbf{0} & 0 & 0 & 3 & 3 & 0 & 913 \\
        \hline
        \tiny \textbf{\kpoly} & 9 & \textbf{0} & 9 & 5 & 4 & 2 &  922 \\ 
        \hline
        \tiny \textbf{\deepsrgr} & 2 & 2 & \textbf{0} & 3 & 3 & 0 & 915 \\ 
        \hline
        \tiny \textbf{\alphabeta} & 250 & 243 & 250 & \textbf{0} & 1 & 163 & 1160 \\ 
        \hline
        \tiny \textbf{\ovaltool} & 258 & 259 & 258 & 9 & \textbf{0} & 170 & 1168 \\
        \hline
        \rowcolor{green!20}
        \tiny \textbf{\drefine} & 170 & 163 & 168 & 76 & 75 & \textbf{0} & 1073 \\
        \hline
    \end{tabular}
    % \caption{Pairwise comparison of tools on adversarially trained networks, e.g. entry on row \kpoly{} and column \deeppoly{} represents 9 benchmark instances on which \kpoly{} verified but \deeppoly{} fails. The green row highlights the number of solved benchmark instances by \drefine{} and not others while the red column is the opposite.}
    \caption{Pairwise comparison of tools on adversarially trained networks}
    \label{tb:matrix2}
\end{figure}





% \subsubsection{Pullback approach: }

% Suppose the abstractCEX is $\boldsymbol{v_0}, \boldsymbol{v_1}, ... , \boldsymbol{v_k}$. 
% The core idea of this approach is to find a point $\boldsymbol{p_{k-1}}$ in the layer $l_{k-1}$. 
% The point $\boldsymbol{v_k}$ is guaranteed to be reachable from point $\boldsymbol{p_{k-1}}$ in the concrete domain.
% Similarly, we find points $\boldsymbol{p_{k-3}}, \boldsymbol{p_{k-5}}, ... \boldsymbol{p_0}$, 
% such that $p_i$ is always reachable from point $p_{i-2}$. 
% We find these points on each $\relu${} layer and input layer only. 
% If we find the point $\boldsymbol{p_0}$ in the input layer, it means $\boldsymbol{p_0}$ is a counter-example, 
% because we can reach from $\boldsymbol{p_0}$ to $\boldsymbol{p_2}$, $\boldsymbol{p_2}$ to $\boldsymbol{p_4}$ 
% , and so on up to $\boldsymbol{v_k}$. 
% If we get stuck in some layer $l_i$ i.e. fails to find point 
% $\boldsymbol{p_{i-2}}$. It means point $\boldsymbol{p_i}$ does not have its corresponding point $\boldsymbol{p_{i-2}}$, 
% which implies that point $\boldsymbol{p_i}$ is a spurious point generated by $\relu${} layer $l_i$. 
% The algorithm \ref{algo:refine1} compute such points. In line number $2$, we equate the value of 
% each element of $\boldsymbol{v_k}$ to the corresponding neuron's affine expression($lexpr$ or $uexpr$), 
% and take the conjunction, and check satisfiability. Since the affine expression of each neuron in $l_k$ contains the 
% variable of layer $l_{i-1}$, so, the satisfying assignment is the point $p_{k-1}$. Similarly, we build the constraints
% for each hidden affine layer's neurons. For a neuron $n_{ij}$ of affine layer $l_i$, 
% if the corresponding point's value $p_{i+1}(j)$ is greater than $0$ then we equate the affine expression of $x_{ij}$ to $p_{i+1}(j)$,
% otherwise, we set the lower and upper bound of the affine expression of $x_{ij}$ as $A(x_{ij}).lb$ and $0$ respectively. 
% Which is the replication of $\relu${} function $x_{(i+1)j} = max(0, x_{ij})$. We construct such constraint 
% for each neuron of $l_i$, and build a formula by taking the conjunction
% of each neuron’s constraint and checking the satisfiability of this formula.
% If it is satisfiable then the point $\boldsymbol{p_{i-1}}$ is found, 
% otherwise, we get the $\mathtt{unsat}$core. We collect all the neurons of $l_i$ whose constraints are 
% in $\mathtt{unsat}$core, and return them as the markedNeurons.   



% \begin{algorithm}[t]
%     \textbf{Name: } pullback \\
%     \textbf{Input: } $\langle N,P,Q \rangle$, abstract constraints $A$ and abstractCEX $=\boldsymbol{v_0}, \boldsymbol{v_1}, .., \boldsymbol{v_k}$ \\
%     \textbf{Output: } markedNeurons or cex. 
%     \begin{algorithmic}[1]
%      \State \textbf{return} $\boldsymbol{v_0}$ if $N(\boldsymbol{v_0}) \models \neg Q$. 
%      \State $constr := \Land_{j=1}^{|l_k|} (A(x_{kj}).lexpr = v_k(j))$
%      \State isSat = checkSat(constr)
%      \If{isSat} \Comment{must be true, last affine layer dont add spurious information}
%         \State $\boldsymbol{p_{k-1}} = \boldsymbol{satval_{k-1}}$ 
%      \EndIf
%      \For{$i=k-1$ to $1$}
%         \If{$l_i$ is affine layer}
%           \State $layerConstraints := true$
%           \For{$j=1$ to $|l_i|$}
%             \If{$p_{i+1}(j) > 0$}
%               \State $constr_{ij}$ := $(A(x_{ij}).lexpr = p_{i+1}(j)$) \Comment{lexpr=uexpr for affine}
%             \Else
%               \State $constr_{ij}$ := $(A(x_{ij}).lb \leq A(n_{ij}).lexpr \leq 0$)
%             \EndIf
%             \State $layerConstrains := layerConstrains \land constr_{ij}$
%           \EndFor
%           \State isSat = checkSat(layerConstrains)
%           \If{not isSat}
%             \State markedNeurons = \{$n_{ij}$ | $1 \leq j\leq |l_i| \land constr_{ij}$ $\in$ unsatCore\}
%             \State \textbf{return } markedNeurons
%           \Else
%             \State $\boldsymbol{p_{i-1}} = \boldsymbol{satval_{i-1}}$
%           \EndIf
%         \EndIf
%      \EndFor
%       \State \textbf{return} $\boldsymbol{p_0}$ \Comment{cex if pullbacked till input layer}
%     \end{algorithmic}
%     \caption{A pullback approach to get mark neurons or counter example}
%     \label{algo:refine1}
%   \end{algorithm}



% \subsection{Utilizing of spurious information}
% The \emph{isVerified} function in algorithm~\ref{algo:main} calls either algorithm~\ref{algo:verif1} or algorithm~\ref{algo:verif2}. 
% Both algorithms~\ref{algo:verif1} and \ref{algo:verif2} take \markednewrons{} as input. 
% The \markednewrons{} represent the set of the culprit neurons. 
% Algorithm~\ref{algo:verif2} splits each neuron of \markednewrons{} into two sub-cases. Suppose the 
% neuron $x\in [lb,ub]$ belongs to \markednewrons{}, then the first case is when $x \in [lb,0]$, and the second case is when $x \in [0,ub]$.   
% After splitting neurons into two cases \deeppoly{} runs for both cases separately. In algorithm \ref{algo:verif2}, \deeppoly{}
% runs an exponential number of times in the size of the \markednewrons{}. We return verified in algorithm~\ref{algo:verif2} if 
% the verification query is verified in all the \deeppoly{} runs. If \deeppoly{} fails to verify in any case then we return the abstractCEX. 

% \begin{algorithm}[t]
%   \textbf{Name: } isVerified2 \\
%   \textbf{Input: } $\langle N,P,Q \rangle$ with abstract constraints $A$ and markedNeurons \\
%   \textbf{Output: } verified or  abstractCEX. 
%   \begin{algorithmic}[1]
%     \For{all combination in $2^{markedNeurons}$}
%       \State $A'$ = run deeppoly
%       \State $constr := P \land (\Land_{i=1}^k lc'_i) \land \neg Q$ \Comment{$lc'$ is with respect to $A'$}
%       \State isSat = checkSat(constr)
%       \If{isSat}
%         \State \textbf{return} abstractCEX
%       \EndIf
%     \EndFor
%     \State \textbf{return} verified
%   \end{algorithmic}
%   \caption{An approach to verify $\langle N,P,Q \rangle$ with abstraction A}
%   \label{algo:verif2}
% \end{algorithm}




% \textbf{Pullback approach}\\
% % The core idea of this approach is to find a point in the predecessor layer, such that the current point 
% % can be reached from the predecessor layers point. 
% This approach finds a point of $x_7=2,x_8=1$ in the predecessor layer means the value of $x_5$ and $x_6$. 
% The point $x_7=2$ and $x_8=1$ must reach from the values of $x_5$ and $x_6$. 
% We compute the value by using the \sat{} query on the affine layer. The \sat{} query shown 
% in equation~\ref{eq:back1}, 
% build by using the values and affine expression of $x_7$, $x_8$, and by using the bounds of $x_5$ and $x_6$.
% The satisfying value of equation~\ref{eq:back1} is $x_5=2$ and $x_6=0$, which is the predecessor point of $x_7=2$
% and $x_8=1$.  
% \begin{equation}
%     \begin{aligned}
%         x_7 = 2 & \land x_8 = 1 \\
%         x_5 = 2 & \land x_6 + 1 = 1 \\ 
%         x_5=2\land x_6+1 = 1 & \land 0\leq x_5 \leq 2 \land 0\leq x_6 \leq 2 \\
%     \end{aligned}
% \label{eq:back1}
% \end{equation}

% In this approach, we compute the predecessor point by considering both the \texttt{affine} and \texttt{relu} layers together. 
% We compute the predecessor point of $x_5=2, x_6=0$ in terms of $x_1$ and $x_2$.
% Since $x_6=0$ and $x_6=max(0,x_4)$ then $x_4$ can be anything from its lower bound to $0$ i.e. $-2 \leq x_4 \leq 0$.
% Similarly the value of $x_5=2$ and $x_5=max(0,x_3)$ then the value of $x_3$ will be $2$. 
% The \sat{} query build as shown in equation~\ref{eq:back2}. 

% \begin{equation}
%     \begin{aligned}
%         x_5 = 2 & \land x_6 = 0 \\
%         x_3 = 2 & \land -2\leq x_4 \leq 0 \\ 
%         x_1+x_2=2\land -2\leq  x_1+x_2 \leq 0 & \land -1\leq x_1 \leq 1 \land -1\leq x_2 \leq 1 \\
%     \end{aligned}
% \label{eq:back2}
% \end{equation}

% When we check the satisfiability of equation~\ref{eq:back2}, it returns \unsat{}. It means the predecessor point of 
% $x_5=2,x_6=0$ does not exist, which implies that point $x_5=2, x_6=0$ is introduced by the triangle approximation.
% Whenever the solver returns \unsat{}, we also get the \unsatcore{}~\ref{sec:solver}. In this example we get the 
% \unsatcore{}=$\{x_1+x_2=2, -2\leq  x_1+x_2 \leq 0\}$, so, we mark the corresponding neurons $x_3$ and $x_4$ as a marked neurons. 
% And refine the marked neurons in the refinement procedure. 


% \textbf{Refinement} \\
% We have two approaches for the refinement {\em Splitting based approach} and {\em MILP-based approach}.
% We elaborate here the {\em MILP-based approach} only. The other approach can be seen in algorithm~\ref{algo:verif2}. 
% These two approaches refine the marked neurons. We got two different sets of marked neurons in the above analysis. 
% The {\em pullback} approach returns $\{x_5, x_6\}$ as the marked neurons, while {\em optimization-based approach}
% return $\{x_6\}$ as the marked neurons. In the following analysis, we are taking the marked neurons set as $\{x_5, x_6\}$.
% Although the analysis will also work if we take the other set of marked neurons.

% In {\em MILP based approach}, we add the exact encoding of the marked neuron ($x_5, x_6$) in addition to the constraints
% in equation~\ref{eq:conjunction} and check the satisfiability, now it becomes \unsat{}, hence, the property verified. 
% The exact constraint of a $\relu${} neuron is explained in equation~\ref{eq:reluexact}. 

% The {\em Splitting based approach} splits the bounds at $0$ of the incoming neurons of the marked neurons. 
% The incoming neurons of $x_5$ and $x_6$ are $x_3$ and $x_4$ respectively. 
% The affine expression of both the incoming neurons is $x_1+x_2$. 
% When we split both $x_3$ and $x_4$ then four cases get generated. The property should get verified in all four cases. 
% All the cases are explained as follows: 
% % Suppose I split $x_3$ then two cases generated
% % $x_3 > 0$ and $x_3 \leq 0$, which implies $x_1+x_2 > 0$ and $x_1+x_2 \leq 0$ respectively. 
% % When we split both $x_3$ and $x_4$ then four cases get generated. Out of four cases $x_3 > 0 \land x_4 \leq 0$ and 
% % $x_3 \leq 0 \land x_4 > 0$ are infeasible cases, because the corresponding affine expression 
% % $x_1+x_2 > 0 \land x_1 + x_2 \leq 0$ is infeasible. We analyze the remaining tow cases as follow: 
% \begin{itemize}
%     \item $x_3 > 0$ and $x_4 > 0$, since neurons are in active phase, so, upper and lower expressions for $x_5$ 
%     becomes $x_3$, similarly the upper and lower expression for $x_6$ becomes $x_4$. 
%     The expression for $x_7$ and $x_8$ remains the same. The upper bound of $x_7 - x_8$ computed in equation~\ref{eq:split1} is $-1$. 
%     The property got verified.   
%     \item $x_3 \leq 0$ and $x_4 \leq 0$, since both neurons are in passive phase, so, the upper and lower expressions 
%         for both the neurons become $0$. The upper and lower expressions of the other neurons remain same. 
%         The upper bound of $x_7 - x_8$ is computed $-1$ in equation~\ref{eq:split2}. The property got verified here too. 
%     \item $x_3 \leq 0$ and $x_4 > 0$, is infeasible case because $x_1+x_2 \leq 0 \land x_1 + x_2 > 0$ is infeasible. 
%     \item $x_3 > 0$ and $x_4 \leq 0$, is infeasible case because $x_1+x_2 > 0 \land x_1 + x_2 \leq 0$ is infeasible. 
% \end{itemize}

% \begin{equation}
%     \begin{aligned}
%         x_7 - x_8 \leq & x_5 - x_6 - 1 \\
%        \leq & x_3 - x_4 -1 \\
%        \leq & x_1 + x_2 -(x_1+x_2) -1 \\
%        \leq & -1 
%     \end{aligned}
%     \label{eq:split1}
% \end{equation}

% \begin{equation}
%     \begin{aligned}
%       x_7 - x_8 \leq & x_5 - x_6 - 1 \\
%         \leq & 0 - 0 - 1 \\
%         \leq & -1
%     \end{aligned}
%     \label{eq:split2}
% \end{equation}




\end{document}

% --- do not erase below this line ----

%%% Local Variables:
%%% mode: latex
%%% TeX-master: t
%%% End:
