\begin{figure}[t]
    \centering
    \begin{tikzpicture}[level/.style={sibling distance=60mm/#1}]
        \node(g){$\globally$} child {node (u) {$\until$} child {node (s) {$\varphi(2)$} child[dashed, gray]
            {node[draw, circle, minimum size=0.3cm, gray] (l) {} child
            {node[draw, circle, minimum size=0.3cm, gray] (ll) {}} child
            {node[draw, circle, minimum size=0.3cm, gray] (lr) {}}}
            child[dashed, gray] {node[draw, circle, minimum size=0.3cm] (r) {}
            child {node[draw, circle, minimum size=0.3cm, gray] (rl)
            {}} child {node[draw, circle, minimum size=0.3cm, gray]
            (rr) {}}}} child {node
            (x) {$x$}}};
    \end{tikzpicture}
    \caption{Example subtree construction for Hybrid Pattern Matching}
\label{fig:subtree}
\end{figure}

We have presented a scheme of ranking formulae and an algorithm to find the best
formula ranked on a trace.
%
However, our scheme has many possibilities of variations. For example, the
choice of various constants and relative importance of LTL operators is one
such. 
%
In this section, we will discuss a couple of variations.
% \sankalp{Possibly an example involving several equally scored formulae? Thread
% locking? th1\_wait U $\neg$ lock, as well as th2\_wait U $\neg$ lock?}

\emph{Hybrid Pattern Matching: }
The algorithm as defined in  sections \ref{sec:cso} and \ref{sec:opm} suffers
from a certain common drawback, though at different ends of the spectrum.
%
Otherwise, we can always discover more and more complex formulae.
%
The issue is further enhanced by the constraint sizes growing far too  quickly
with increase in the depth of the pattern. 

In the case of  \emph{pattern matching}, the algorithm does not check a large
enough variety of properties (it is guided by the template pattern) and, for the
same reason,  is also quite \emph{insufficiently} expressive in comparison with
constrained system optimization. 
%
To remedy this, we allow a middle ground, where, instead of attempting to learn
formulae from scratch or from explicit patterns, we allow learning
\emph{subformulae} within some known patterns.
%
A subformulae argument $\varphi(n)$ with $n$ being a prescribed maximum depth for
the subtree can be provided as part of the pattern, parsed into the tree as an
abstracted empty formula with constraints constructed for the specified nodes
explicitly, and for the subformulae recursively in the manner as described for
seciton \ref{sec:cso}. In Figure \ref{fig:subtree}, we show an example of the
hybrid pattern $\globally (\varphi(2) \: \until \: x)$, where $\varphi(2)$ is an
unknown formula of depth less than $2$ and $x$ is an unknown letter.

\begin{algorithm}
    \begin{algorithmic}[1]
        \Procedure{HybridPattern}{$P, N, pattern$} \Comment{Returns optimal
            formula fitting a pattern}
        
            \State \textbf{parse} pattern 
            \State construct $\varphi^{\var}_\prop$ for propositional patterns in tree
            \State construct $\varphi^{ST}_\textnormal{n}$ for every subformula pattern $\varphi(n)$ in the tree
            \State constraint $\Phi \leftarrow \varphi^{\var}_\prop \land \bigwedge \varphi^{ST}_\textnormal{n}$
            \State optimize
            $\min( \{y_{0,0}^\tau\: |\: \tau \in P\} )$ with $\Phi$ as constraint
            
            \If{optimization succeeds with model m} \Comment SAT \State
                construct formula tree from m \State \textbf{return}
                optimized formula tree \Else \State \textbf{return} UNSAT \EndIf
                \EndProcedure
    \end{algorithmic}
    \caption{Computing the optimal formula given a partial pattern}
    \label{algo:hybrid}
\end{algorithm}

\emph{Prioritize variables: }
%
In case there is a large set of events and we want bias the focus of our search
towards certain letters in the traces that do not occur very often, we may
adjust the value $V(p,w)$ assigned to each propositional variable $p$.
%
In our default scheme, we assign $V(p,w) = 1$ if $w(1)$ contains $p$.
%
A user may assign a value greater than $1$ to variables $p$ that are desirable
and assign less than $1$ for the variables $p$ that are not.
% %
% As a variation of Optimized Pattern Matching, we let the user to fix a subset
% of  variables to appear in the pattern and compute the best mapping of the
% other variables per specification.
%
This allows for mining specifications pertaining to the prioritized variables in
cases where several competing well ranked specifications are present.
%
Let $\pi$ be the map from the propositional variables $\prop$ to their priority.
%
We replace equation~\eqref{eq:prop_score} by the following formula where we
return score $\pi(p)$ instead of $1$. 
% %
% To achieve this, in addition to~\eqref{eq:map}, the following
% constraint~\eqref{eq:fixed} is added, which binds the common variable to
% itself.
% %
% \begin{align} \bigwedge _{p \in Var \intersection \prop} m_{p, p}
%     \label{eq:fixed} \end{align}

\begin{align}
    \bigwedge _{1 \leq i \leq N} \bigwedge _{p \in \prop} x_{i, p} \rightarrow \left[ 
        \bigwedge _{1 \leq t \leq |\tau|} y^\tau _{i, t} = \begin{cases}
            \pi(p) & \textnormal{if } p \in \tau (t) \\
            0 & \textnormal{if } p \not \in \tau (t)
        \end{cases}
    \right] \label{eq:prop_priority_score} 
\end{align}

Recall that we considered the trace in figure~\ref{fig:exampletrace}, where we
wanted to find a property $\globally \:\texttt{auth\_failed}\: \Leftrightarrow\:
\varphi(1)$. We may further abstract the property $\globally \: \varphi(0) \:
\Leftrightarrow\: \varphi(1)$ by setting the priority map $\pi =
\{\texttt{auth\_failed} \mapsto 10, otherwise \mapsto 1\}$. We obtain the same
result from \ourtool.

% Of course, the best mapping for $x$ is $\texttt{connected}$ and the tool
% returns the formula $\globally (\texttt{connected} \until
% \texttt{disconnected})$. 

The above variations suggest that our method is viable to be adapted  to an
application at hand, where we want to bias our ranking to give preference to a
desired class of formulae.


% --- no delete below this line --

%%% Local Variables:
%%% mode: latex
%%% TeX-master: "main"
%%% End:
